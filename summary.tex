

\title{Schlüsselexperimente in der Teilchenphysik}
\author{
        Jean-Marco Alameddine \\
}
\date{\today}

\documentclass[12pt]{article}

% deutsche Spracheinstellungen
\usepackage{polyglossia}
\setmainlanguage{german}

\usepackage{suffix}

\usepackage{geometry}
\geometry{a4paper,left=25mm,right=25mm, top=3cm, bottom=3cm}

% unverzichtbare Mathe-Befehle
\usepackage{amsmath}
% viele Mathe-Symbole
\usepackage{amssymb}
% Erweiterungen für amsmath
\usepackage{mathtools}
\usepackage{physics}
% Fonteinstellungen
\usepackage{fontspec}
% Latin Modern Fonts werden automatisch geladen

\usepackage[
  math-style=ISO,    % ┐
  bold-style=ISO,    % │
  sans-style=italic, % │ ISO-Standard folgen
  nabla=upright,     % │
  partial=upright,   % ┘
  warnings-off={           % ┐
    mathtools-colon,       % │ unnötige Warnungen ausschalten
    mathtools-overbracket, % │
  },                       % ┘
]{unicode-math}

% traditionelle Fonts für Mathematik
\setmathfont{Latin Modern Math}
\setmathfont{XITS Math}[range={scr, bfscr}]
\setmathfont{XITS Math}[range={cal, bfcal}, StylisticSet=1]

% Zahlen und Einheiten
\usepackage[
  locale=DE,                 % deutsche Einstellungen
  separate-uncertainty=true, % immer Fehler mit \pm
  per-mode=reciprocal,       % ^-1 für inverse Einheiten
  % alternativ:
  % per-mode=reciprocal, % m s^{-1}
  % decimal-marker=., % . statt , f�r Dezimalzahlen
]{siunitx}

% chemische Formeln
\usepackage[
  version=4,
  math-greek=default, % ┐ mit unicode-math zusammenarbeiten
  text-greek=default, % ┘
]{mhchem}

% Grafiken können eingebunden werden
\usepackage{graphicx}
% größere Variation von Dateinamen möglich (Probleme mit Leerzeichen behoben)
\usepackage{grffile}

\usepackage{float}

\newcommand\chapterauthor[1]{\authortoc{#1}\printchapterauthor{#1}}
\WithSuffix\newcommand\chapterauthor*[1]{\printchapterauthor{#1}}

\makeatletter
\newcommand{\printchapterauthor}[1]{%
  {\parindent0pt\vspace*{-25pt}%
  \linespread{1.5}\large\scshape#1%
  \par\nobreak\vspace*{35pt}}
  \@afterheading%
}
\newcommand{\authortoc}[1]{%
  \addtocontents{toc}{\vskip-5pt}%
  \addtocontents{toc}{%
    \protect\contentsline{chapter}%
    {\hskip1.3em\mdseries\scshape\protect\scriptsize#1}{}{}}
  \addtocontents{toc}{\vskip5pt}%
}
\makeatother

\usepackage{xfrac}
\usepackage{url}

\begin{document}

\maketitle
\newpage

%\begin{abstract}
%This is the paper's abstract \ldots
%\end{abstract}

\setcounter{tocdepth}{1} %nicht alle Unter-Unterkapitel anzeigen (1 -> Nur Überschriften, 2-> Auch 1. Unterkapitel)

\tableofcontents
\newpage

\section{Cherenkov-Detektoren}

\chapterauthor{Rune Dominik, 26.10.2018}

\subsection{Historische Entwicklung}
Die erste Beobachtung von Cherenkov-Strahlung, welche die Grundlage für Cherenkov-Detektoren bildet, lässt sich auf Marie Curie im Jahre 1910 zurückverfolgen.
Sie beobachtete erstmals das charakteristische blaue Licht, welches im Jahre 1934 durch den späteren Namensgeber Pawel Alexejewitsch Tscherenkow entdeckt wurde.
Er erhielt, zusammen mit Ilja Michailowitsch Frank und Igor Jewgenjewitsch Tamm, welche 1937 den Cherenkov-Effekt theoretisch erklärten, im Jahr 1958 den Nobelpreis.
Seitdem wird die Technik der Cherenkov-Detektoren, welche für den Teilchennachweis und die Teilchenidentifikation in der (Astro-)Teilchenphysik verwendet werden, stetig weiterentwickelt.

\subsection{Theoretische Grundlagen}
Der Cherenkov-Effekt beschreibt das Auftreten von elektromagnetischer Strahlung, wenn überlichtschnelle, geladene Teilchen ein Medium durchqueren.
Besitzt das Teilchen die Geschwindigkeit $\beta c$ und beträgt die Lichtgeschwindigkeit im Medium $\sfrac{c}{n}$, wobei $n$ den Brechungsindex im entsprechenden Medium beschreibt, so gilt für den Winkel zwischen Teilchenbahn und Emission des Cherenkov-Lichts die Formel
\begin{equation}
	\cos{\theta} = \frac{1}{n \beta}.
\end{equation}
Das Spektrum des emittierten Lichts, welches durch die Frank-Tamm-Gleichung beschrieben wird, besitzt hierbei eine $\lambda^{-2}$-Abhängigkeit.

Um Detektoren in ihrer Effizienz vergleichen zu können, kann die Detektoreffizienz $\epsilon$ verwendet werden, welche den Anteil der tatsächlich gemessenen an den theoretisch messbaren Cherenkov-Photonen beschreibt.

\subsection{Experimenteller Aufbau von Cherenkov-Detektoren}
Ein grundlegender Cherenkov-Detektor besteht aus einem Radiator, in welchem durchgehende Teilchen Cherenkov-Licht emittieren sowie einer Detektoreinheit, welche das Licht detektiert. Zudem wird eine Anordnung, beispielsweise aus Spiegeln, benötigt, welche das Licht vom Radiator auf den Detektor lenkt.
Die exakte Ausführung dieses Konzeptes ist detektorabhängig, wobei im Folgenden das Konzept der Schwellwertdetektoren, der differentiellen Detektoren sowie der RICH-Detektoren erläutert.

\subsubsection{Schwellwertdetektoren}
\newpage


\section{Wu-Experiment}

\chapterauthor{Inga Höfmann, 02.11.2018}

\subsection{Historische Entwicklung}
Das Wu-Experiment beschäftigt sich mit der Parität, welche als Operator eine Inversion in allen Raumkoordinaten beschreibt. Es gilt
\begin{align*}
	\hat{P} \vec{r} &= -\vec{r}, &
	\hat{P} \vec{L} &= \vec{L}.
\end{align*}
Die Parität wurde erstmals im Jahre 1927 als Quantenzahl und somit intrinsische Eigenschaft eines Teilchen eingeführt und galt zunächst für alle Wechselwirkungen als erhalten.
Zweifel an der Parität als Erhaltungsgröße kamen im Jahr 1954 mit dem sogenannten $\tau$-$\theta$-Puzzle auf:
Die damals als $\tau$- und $\theta$-Meson bekannten Teilchen besaßen identische Massen, Ladungen und Lebensdauern.
Unterschiede ergaben sich lediglich in den Zerfallsprodukten:
Während das $\tau^+$-Meson in den Endzustand $\pi^+ \pi^+ \pi^-$ (Parität $-1$) zerfiel, war der Endzustand des $\theta^+$-Mesons $\pi^+ \pi^0$ (Parität $+1$).
Als Erklärung dieses Puzzles existierte die Möglichkeit, dass es sich bei beiden Mesonen um das gleiche Teilchen handelt und die Parität in diesem Fall verletzt ist.
Diese Vermutung wurde, zusammen mit Vorschlägen für einen experimentellen Aufbau, im Jahre 1957 von Tsung-Dao Lee und Chen Ning Yang in einem Paper veröffentlicht, wofür sie 1957 den Nobelpreis erhielten.
Das dazugehörige Experiment wurde von der Physikerin Chien-Shiung Wu im Jahre 1956 innerhalb von neun Monaten realisiert.

\subsection{Idee des Wu-Experimentes}
Die Idee des Wu-Experimentes ist es, eine Paritätsverletzung in der schwachen Wechselwirkung nachzuweisen.
Um einen Interferenzterm mit paritätserhaltenden und paritätsverletzenden Anteilen untersuchen zu können muss eine pseudoskalare Messgröße genutzt werden.
Hierzu bietet es sich an, die Winkelverteilung $I\left(\theta \right)$ von $e^-$ in einem $\beta$-Zerfall zu betrachten. 
Es ergibt sich ein Parameter $\alpha$, welcher die Asymmetrie in der Winkelverteilung beschreibt, wobei $\alpha = 0$ für Paritätserhaltung bzw. $\alpha \neq 0$ für Paritätsverletzung gilt.

Das Wu-Experiment ist so konzipiert, dass es rein qualitative Aussagen über den Parameter $\alpha$ treffen kann.
Als Probe wird $\ce{^{60}_{27}Co}$ verwendet, welches dominant unter Aussendung eines $e^-$ und eines $\bar{\nu_e}$ (d.h. unter der schwachen Wechselwirkung) in $\ce{^{60}_{28}Ni^{*}}$ zerfällt. Dieses sendet wiederum zwei energetisch charakteristische Photonen aus um in einen nicht angeregten Zustand zu gelangen.
Zu beachten ist, dass $\ce{^{60}_{27}Co}$ einen Spin von $5$ und $\ce{^{60}_{28}Ni^{*}}$ einen Spin von $4$ besitzt.
Somit müssen die Ausrichtungen der Spin-$z$-Komponente des $e^-$, des $\bar{\nu_e}$ (beide mit Spin $\sfrac{1}{2}$) und des $\ce{^{60}_{28}Ni^{*}}$ alle in die gleiche Richtung zeigen.
Werden die Spins der Kobalt-Atome nun in eine Vorzugsrichtung ausgerichtet, kann die Anzahl der Elektronen in Spinrichtung gemessen werden.
Wird die Vorzugsrichtung der Spins umgekehrt, was einer Paritätstransformation entspricht, so wird die Anzahl der Elektronen gegen die Spinrichtung gemessen.
Somit wird die Messung einer möglichen Paritätsverletzung ermöglicht.

\subsection{Experimentelle Herausforderungen}
Insbesondere die Ausrichtung der Spins der Kobalt-Atome stellt eine große experimentelle Herausforderung dar.
Hierfür werden sowohl sehr geringe Temperaturen ($\approx \SI{1}{\kelvin}$) als auch hohe Magnetfelder ($\approx \SI{10}{\tesla}$) benötigt.
Die Lösung stellte die sogenannte Rose-Gorter-Methode dar.
Hierbei wird ein Kristall, $\ce{CeMg}$-Nitrat, verwendet, wobei das Salz einen anisotropen $g$-Faktor besitzt.
Die starke Abkühlung geschieht durch das Anlegen eines B-Feldes entlang des maximalen $g$-Faktors.
Zunächst werden dabei die, aufgrund der Hyperfeinstrukturaufspaltung entstehenden, niedrigeren Energieniveaus besetzt.
Anschließend wird das System thermisch isoliert und das Magnetfeld heruntergefahren.
Durch diese sogenannte adiabatische Entmagnetisierung können hinreichend geringe Temperaturen erreicht werden.
Das Anlegen eines Magnetfeldes in Richtung des minimalen $g$-Faktors ermöglicht dann die Polarisation der Hüllenelektronen des Kobalts.
Dies führt zu einer starken Zunahme des $B$-Feldes in der Nähe des Kerns und somit zu einer Polarisation der Spins der Kerne.
Werte der Polarisation von bis zu $\SI{60}{\percent}$ konnten hiermit erreicht werden.
Die experimentelle Realisierung ist in Abbildung \ref{fig:wu} dargestellt.

\begin{figure}
  \centering
  \includegraphics[height=7.0cm]{ressources/wu.png}
  \caption{Schematischer Aufbau des Wu-Experimentes \cite{wu}.}
  \label{fig:wu}
\end{figure}

Die Polarisation des Kernspins geschieht über die $\gamma$-Szintillatoren, da die Photonen vorzugsweise in Spin-Richtung emittiert werden.
Der Nachweis der Elektronen geschieht direkt über den Szintillatorkristall über Photomultiplier.
Um die Parität untersuchen zu können wird das gesamte Experiment mit zwei verschiedenen Ausrichtungen des Magnetfeldes und somit mit zwei verschiedenen Vorzugsrichtungen der Spins durchgeführt.

\subsection{Experimentelle Ergebnisse und Konsequenzen}
Experimentell konnte nachgewiesen werden, dass die Kernspins hinreichend polarisiert werden konnten.
Zudem wurde gezeigt, dass die Elektronen vorzugsweise gegen die Richtung des Kernspins des Kobalts emittiert werden.
Dies entspricht einer Verletzung der Parität und bestätigte somit die These von Tsung-Dao Lee und Chen Ning Yang.
Spätere Experimente wiesen quantitativ nach, dass die Parität tatsächlich maximal verletzt ist.
Zudem folgte die theoretische Erkenntnis, dass die schwache Wechselwirkung eine V-A-Struktur besitzt:
Sie koppelt nur an linkshändige Teilchen und rechtshändige Antiteilchen, was die maximale Paritätsverletzung erklärt.

\subsection{Diskussion zum Vortrag}
Als Frage wurde gestellt, ob die Suche nach Paritätsverletzung in anderen Wechselwirkungen ebenfalls durchgeführt wurde.
Hierbei ergab sich die Antwort, dass bis heute nach paritätsverletzenden Anteilen außerhalb der schwachen Wechselwirkung gesucht wird.

Da Kobalt zwei unterschiedliche $\beta^-$-Zerfallsmoden in Nickel besitzt, kam die Frage auf wie diese Unterschiede experimentell berücksichtigt wurden.
Dies war möglich und wurde realisiert, da die Szintillatoren auf die festen Energien der Photonen ausgerichtet werden können.

Zuletzt wurde über technische Details der Kühlung diskutiert.
Hierbei wurde das Pumpen erwähnt, welches genutzt wurde, um eine Kühlung auf bis zu $\SI{1}{\kelvin}$ zu ermöglichen.

\newpage


\section{Die Entdeckung des Gluons}

\chapterauthor{Dominik Hellmann, 09.11.2018}

\subsection{Theoretische Grundlagen der Quantenchromodynamik}
Die Quantenchromodynamik (QCD) ist die Quantenfeldtheorie der starken Wechselwirkung, d.h. der Kraft, die für das Zusammenhalten der Quarks in Hadronen verantwortlich ist.
Grundlage für die QCD ist die Yang-Mills-Theorie, welche eine im Jahre 1954 entwickelte Eichtheorie ist.
Dabei folgt die QCD aus der $SU(3)\text{c}$-Symmetriegruppe, wobei hieraus die drei Farbladungen rot, grün, blau und die dazugehörigen Antifarben resultieren.
Die Farben sind dabei analog zu den elektrischen Ladungen der QED zu verstehen.

Analog zu den Photonen der QED existieren auch Eichbosonen der QCD, welche Gluonen heißen.
Aus der Gruppentheorie folgt, dass 8 Gluonen existieren, welche den Generatoren der $SU(3)_\text{c}$ entsprechen.
Gluonen sind masselose Vektorteilchen, welche jedoch im Kontrast zur QED ebenfalls Farbladung besitzen und dementsprechend an sich selbst koppeln können.
Hieraus folgt die asymptotische Freiheit als charakteristische Eigenschaft der QCD:
Durch Vakuumpolarisationen, dem sogenannten Anti-Screening, wird die Kopplungskonstante der QCD $\alpha_{s}$ klein für große Impulsübertäge. 
Hieraus folgt, dass die einzelnen Quarks bei großen Impulsübertägen, bzw. kleinen Abständen, als asymptotisch frei verstanden werden können.
Andereresits folgt als Charakteristikum das Confinement: Bei großen Abständen steigt das Potential der QCD linear an, d.h. die Kraft für das weitere Vergrößern des Abstandes wächst linear.
Wird der Abstand weiter vergrößtert, so werden neue Teilchen in sogenannten Jets produziert, bis der Endzustand wieder farbneutral ist.
Hierdurch können farbgeladene Teilchen nur in einem farbneutralen, gebundenen Zustand beobachtet werden.
Die entstehenden Jets können in einem Detektor als Ansammlung von Teilchen beobachtet werden - Die Messung der Impulse dieser Teilchen lässt Rückschlüsse auf die Impulse der ursprünglichen Quarks zu.

\subsection{Historische Entwicklung}

Im Jahr 1964 wurden von Murray Gell-Mann erstmals Quarks als Bestandteile der Hadronen postuliert - Hierbei stellen sich die Fragen, welche Kräfte diese Elementarteilchen zusammenhalten und welche Austatuschteilchen diese Kraft vermitteln.
Weitere Hinweise auf eine mit der QCD verbundenen Quantenzahl stellten die $\Delta$-Baryonen sowie der $R$-Plot dar.
So ist die Wellenfunktion des $\Delta^{++}$-Baryons unter Vertauschung von zwei Quarks symmetrisch, falls nur die bisher bekannten Raum-, Flavor- und Spinanteile der Wellenfunktion berücksichtigt werden.
Dies steht jedoch im Widerspruch mit dem Pauliprinzip für Fermionen.
Die Einführung einer Farbe als Quantenzahl stellt eine Lösung dieses Problems dar.
Bei der Betrachtung des $R$-Plots wird das Verhältnis der Produktion von Hadronen aus $e^+ e^-$ Reaktionen in Abhängigkeit der Schwerpunktsenergie betrachtet.
Hierbei ist das Verhältnis $R$ sensitiv auf das Vorhandensein von Farbladungen und sagt experimentell drei Farbladungen vorraus.



% Vorhersage von Jets: 1975 durch Feynman und Beobachtung 1978 am DESY. 1979 wurden 3 Jet Ereignisse beobachtet.
\newpage


\section[Die J/Psi-Entdeckung]{Die $J/\Psi$-Entdeckung}

\chapterauthor{Jan Langer, 16.11.2018}

\subsection{Historische Einordnung und Quarkmodelle}

Die Entdeckung des $J/\Psi$ im Jahre 1974 gilt als Nachweis der Existenz eines 4. Quarks, dem charm-Quark ($c$), sowie als allgemeine Bestätigung der Quark-Theorie.
Es handelt sich dabei um ein flavour-neutrales Meson, welches aus einem $c$ sowie einem $\overline{c}$ besteht.
Bereits im Jahr 1961 wurde der sogenannte Eightfold Way (Achtfache Weg) von Gell-Mann und Zweig vorgeschlagen, um die Teilchen mit der 1953 eingeführten Quantenzahl der Strangeness systematisch anzuordnen.
Hierbei werden die Mesonen, welche aus den zu diesem Zeitpunkt bekannten Quarks $u$, $d$ und $s$ bestehen, in einem Oktett sowie die Baryonen in einem Dekuplett angeordnet.
Dies entspricht einer Flavour $SU(3)$-Symmetrie, die annähernd zwischen den Quarks gültig ist.
Die Entdeckung des anhand diesem Modell postulierten $\Omega^-$-Quarks im Jahre 1964 festigte diese Theorie.


Im Jahr 1970 trat ein theoretisch unerwartetes Messergebnis im Zerfall von $K^0_\text{L}$-Mesonen auf, welches den sogenannten GIM-Mechanismus motivierte:
So konnte es bisher nicht erklärt werden, wieso der experimentell bestimmte Wert
\begin{align*}
	\frac{\Gamma\left( K^0_\text{L} \to \mu^+ \mu^- \right)}{\Gamma\left( K^0_\text{L}  \to \text{all}\right)} = \left(\num{9.1+-1.9}\right) \cdot \num{e-9}
\end{align*}
so gering ist.
Der von Glashow, Iliopoulus und Mainani beschriebe GIM-Mechanimus erklärt die Unterdrückung dieses Zerfalles durch eine destruktive Interferenz in Loop-Diagrammen erster Ordnung. Diese tritt durch die Einführung eines vierten Quarks, dem $c$-Quarks, auf.

Der Nachweis des $c$-Quarks über das $J/\Psi$ im Jahre 1974 bestätigt diese theoretische Erklärung, die bereits 1973 zur Erklärung der CP-Verletzung auf 3 Quarksfamilien erweitert wurde.

\subsection{Die Experimente am SLAC und BNL}
Das $J/\Psi$ wurde durch zwei unabhängige Experimente, dem Stanford Linear Accelerator Center (SLAC) und dem Brookhaven National Laboratory (BNL), entdeckt.
Die Entdeckung wurde auf einer gemeinsamen Pressekonferenz am 11. November 1974 offiziell bekannt gegeben.

Das Experiment am BNL, welches von Samuel Chao Chung Ting geleitet wurde, war ein fixed-target Experiment zur Suche nach Vektor-Mesonen und zur Untersuchung von deren Eigenschaften.
Der Beschleuniger konnte Protonen auf eine Energie von \SI{33}{\giga\electronvolt} beschleunigen, der Detektor war ein Zwei-Arm-Spektrometer.
Das Experiment am SLAC hingegen, unter Leitung von Burton Richter, war ein $e^+ e^-$-Speicherring mit dem Ziel, Hadronproduktion bei möglichst genauer Energie zu untersuchen.
Hierbei sollte insbesondere das Verhältnis $R$, d.h. der Anteil der Hadronproduktion bei Elektron-Positron Kollision, gemessen werden.
Der genutzte Detektor Mark1 war ein $4\pi$-Detektor, die maximale Schwerpunktsenergie im Schwerpunktssystem betrug $\SI{8}{\giga\electronvolt}$. 

\subsection{Ergebnisse der Experimente}

Der Peak in den Daten bei ca. $\SI{3.1}{\giga\electronvolt}$, welches der $J/\Psi$-Resonanz entspricht, wurde zunächst im August 1974 am BNL beobachtet, wobei zunächst auf eine Publikation verzichtet wurde.
Da es sich um einen $pp$-Collider handelt, war mussten bei den Analysen zur Trennung von Signal und Untergrund Spurrekonstruktionen durchgeführt werden.
Zudem wurden bei der Analyse der Daten  viele Gegenproben durchgeführt, beispielsweise durch die Änderung der Magnetströme, der Strahlintensität oder der Targetdicke.

Anfang November entdeckte auch das SLAC eine Erhöhung des Wirkungsquerschnittes $\sigma\left(e^+ e^- \to \text{hadrons}\right)$ im passenden Energiebereich, nachdem die Energie des Colliders auf die Resonanzenergie zurückgefahren wurde.
Feinere Messschritte ergaben eine deutliche Erhöhung des Wirkungsquerschnittes für hadronische Endzustände und somit den Nachweis der $J/\Psi$-Resonanz.

Während das SLAC-Experiment eine obere Grenze der Zerfallsbreite von $\Gamma \leq \SI{1.3}{\mega\electronvolt}$ angeben konnte, waren die Ergebnisse von BNL mit $\Gamma = 0$ kompatibel.
Die mit der schmalen Zerfallsbreits verbundene hohe Lebensdauer konnte mit den bisherigen Quarks nicht erklärt werden.
Die Deutung Tings, dass das $J/\Psi$ dementsprechend aus der Kombination $c\overline{c}$ besteht, bestätigte sich bereits nach einem Jahr.

Im Jahre 1976 erhielten beide Experimentatoren den Nobelpreis für die Entdeckung des $c$-Quarks.
Bereits wenige Tage nach der $J/\Psi$-Entdeckung folgte die Entdeckung vieler weiterer Resonanzen - diese Zeit wird historisch auch als Novemberrevolution bezeichnet.

\subsection{Diskussion}
In der Diskussion wurde gefragt, wie genau es zu der Novemberrevolution nach der $J/\Psi$-Entdeckung gekommen ist.
Nach der Publikation wurde genauer in den entsprechenden Energiebereichen gesucht, zusätzlich sind die Experimente im Laufe der Zeit besser geworden.

Zudem wurde der qualitative Unterschied zwischen einer $e^+ e^-$-Maschine und einem $pp$-Experiment besprochen.
Bei einem $e^+ e^-$ Collider handelt es sich um ein Präzisions-Experiment: Die Reaktionen sind sehr sauber und können durch einfaches Zählen der Endprodukte gut verstanden werden.
Dafür ist bei diesen Experimenten nur eine feste Schwerpunktsenerige vorhanden, die für neue Entdeckungen fortlaufend verändert werden muss.
Bei $pp$-Experimenten müssen die Endzustände aufgrund der hadronischen Reaktionen aufwendig rekonstruiert werden.
Dafür kann, aufgrund der an den Reaktion beteiligten Partonen, gleichzeitig ein großes Energiespektrum untersucht werden.
\newpage



\section{Entdeckung von Neutrinos}

\chapterauthor{Felix Geyer, 23.11.2018}

\subsection{Historische Postulierung des Neutrinos}
Historisch betrachtet geht die erste Postulierung des Neutrinos auf die Betrachtung des $\beta^-$-Zerfalles um das Jahr 1930 zurück.
Zu dieser Zeit wurde der $\beta^-$-Prozess als Zweikörperzerfall $\ce{n -> p + e^-}$ betrachtet, was kinematisch zu einer diskreten Energie des Elektrons führt.
Entgegen dieser Vorhersage wurde jedoch ein kontinuierliches Energiespektrum für das Elektron gemessen.
Zusätzlich wird die Drehimpulserhaltung bei dieser Betrachtung des Zerfalles verletzt.

Dies führte zu der ersten Vorhersage eines Neutrinos durch Pauli, der ein sehr leichtes Teilchen mit halbzahligen Spin vorhersagte, welches beim $\beta^-$-Zerfall zusätzlich zum Proton und Elektron entsteht.
Bei einem Dreikörperzerfall wäre das gemessene kontinuierliche Energiespektrum des Elektrons erklärbar.
Paulis Theorie zum $\beta^-$-Zerfall wurde 1934 erstmals publiziert.

\subsection{Experimenteller Nachweis des Neutrinos}
Im Laufe der Zeit existierten mehrere verschiedene Ideen und Experimente, um einen Nachweis von Neutrinos zu ermöglichen.
Führend bei der Entwicklung dieser Ideen und Experimente waren Frederick Reines und Clyde Cowan.

Die grundlegende Idee bestand darin, Neutrinos durch den Neutroneneinfang $\overline{\nu_e} + p \rightarrow n + e^+$ nachzuweisen.
Um eine große Neutrinoquelle zu erhalten und somit den vorhandenen, großen Untergrund zu eliminieren sollte eine Atombombe gezündet werden.
Hierbei spielte auch die Tatsache eine Rolle, dass der theoretisch berechnete Wirkungsquerschnitt dieser Nachweisreaktion bereits kurz nach der Veröffentlichung von Fermis Theorie zum $\beta$-Zerfall zu $\approx \SI{1e-20}{\barn}$ berechnet wurde, d.h. die erwartete Reaktion war äußerst selten.
Als Detektor sollte ein flüssiger Szintillationsdetektor verwendet werden, welcher auch auf größeren Skalen genutzt werden kann.
Durch Paarvernichtung des Positrons aus dem $\beta^-$-Zerfall mit vorhandenen Elektronen entstehen zwei Photonen, die im Szintillationsdetektor Lichtblitze verursachen.
Diese können nun durch Photomultiplier detektiert werden.

Um auf den Aufwand der Zündung einer Atombombe verzichten zu können, und somit auch eine einfachere Reproduzierbarkeit des Experimentes zu ermöglichen, entstand eine verbesserte Idee für den experimentellen Aufbau:
Hierbei sollte auch das beim $\beta^-$-Zerfall entstehende Neutron nachgewiesen werden indem die beim Kerneinfang entstehende $\gamma$-Strahlung im Detektor nachgewiesen wird.
Dafür wurde mit Cadmium versetztes Wasser als Reaktionsmedium verwendet.
Der verzögerte Nachweis von Positron und Neutron, d.h. die Koinzidenz, wäre eine eindeutige Signatur für Neutrinos und ermöglicht eine starke Reduzierung von Untergrundereignissen.
Diese Verringerung des Untergrundes erlaubte es einen Kernreaktor als Neutrinoquelle zu verwenden.
Der Detektor wurde dabei knapp $\SI{2}{\metre}$ vom Reaktor entfernt aufgestellt.
Um die direkt vom Reaktor erzeugten Neutronen abzuschirmen wurde der Detektor in eine Borax-Wasser-Lösung getaucht.
Zudem wurde eine Anti-Koinzidenz für den Ausschluss kosmischer Strahlung sowie Quecksilber und Blei zur Abschirmung gegen natürliche Radioaktivität verwendet.
Die Messungen wurden insbesondere durch einen nach wie vor zu hohen Untergrund durch kosmische Strahlung sowie elektronisches Rauschen, vor allem durch die vielen Photomultiplier, erschwert.
Das Experiment wurde sowohl bei eingeschaltetem als auch bei ausgeschaltetem Reaktor durchgeführt, wobei auch in letzterem Fall eine Signalrate nachgewiesen wurde.
Dieser Hintergrund konnte später als kosmische Strahlung identifiziert werden.

Um eine aussagekräftige Messung zu erhalten wurde das Experiment im November 1955 an einem Reaktor im Savannah River Plant Area neu aufgebaut.
Einerseits besaß dieser Reaktor eine hohe Reaktorleistung und somit einen hohen erwarteten Antineutrinofluss, zudem befand sich der Detektor $\SI{12}{\metre}$ unter der Erde, was eine bessere Abschirmung von komischer Strahlung ermöglichte.
Als Reaktionsmedium wurde ein mit Cadmium versetzter Wassertank verwendet, welcher oben und unten von zwei Szintillatoren umgeben war.
Die Messungen wurden über 100 Tage durchgeführt, die ermittelte Signalrate betrug $\SI{3.0+-0.2}{\per\hour}$.
Um nachzuweisen, dass der erste Impuls tatsächlich durch die Positronen auftritt, wurde testweise eine Bleischicht zwischen dem Wassertank und einer der Szintillatoren aufgestellt, was zu einer geringeren Detektion von ersten Impulsen in diesem Szintillator führt.
Zur Kontrolle des durch Neutronen auftretenden zweiten Impulses wurde das Cadmium aus dem Wassertank entfernt, wodurch die Rate komplett verschwindet da kein Neutroneneinfang mehr stattfindet.
Die Resultate dieses Experimentes gelten somit als Entdeckung des Neutrinos.
Reines erhielt 1995 den Nobelpreis für seine Arbeiten im Bereich der Neutrinophysik.

\subsection{Weitere Entdeckungen und Experimente der Neutrinophysik}
Während beim Savannah River Experiment Elektronneutrinos untersucht wurden, konnte 1962 von Leon Lederman, Melvin Schwatz und Jack Steinberger am Alternating Gradient Synchrotron auch das Myonneutrino nachgewiesen werden.
Hierfür wurden Pionen mithilfe eines Fixed-Target Experimentes erzeugt.
Diese zerfielen in Myonen sowie Myonneutrinos, wobei der entstehende Teilchenstrahl von einer Stahlwand abgeschirmt wurde.
Im Gegensatz zu den Myonen können die Neutrinos diese Wand jedoch assieren und durch Kerneinfang als entstehende Myonen nachgewiesen werden.

Im Jahre 2000 wurde durch das DONUT-Experiment am Fermilab ebenfalls das Tauneutrino nachgewiesen.
Erschwerend bei dessen Nachweis ist die geringe Lebensdauer der entstehenden Tauonen beim Einfang von Neutrinos durch Neutronen.

Heutige betriebene Neutrinoexperimente sind beispielsweise ANTARES, IceCube oder (Super-)Kamiokande.
Diese Experimente dienen verschiedenen Zwecken wie dem Nachweis von atmosphärischen, solaren oder kosmischen Neutrinos der unterschiedlichen Neutrinofamilien.
\newpage


\section{Neutrinooszillationen}

\chapterauthor{Hubertus Kaiser, 30.11.2018}

\subsection{Das solare und das atmosphärische Neutrinoproblem}

In den 1960er Jahren trat mit dem solaren Neutrinoproblem erstmals eine Beobachtung auf, welche nicht ohne Probleme mit den bisher im Standardmodell als masselos angenommenen Neutrinos gelöst werden konnte.
Heutzutage ist bekannt, dass Neutrinos in drei Generationen ($\nu_e, \nu_\mu, \nu_\tau$) auftreten.
Bereits früh war bekannt, dass die Sonne bei seiner Energiegewinnung Elektronneutrinos produziert.
Mit dem Homestakeexperiment, welches 1970 mit seiner Messung begann, wurde die Anzahl an Neutrinos aus der Sonne gemessen.
Hierzu befand sich $\SI{1.5}{\kilo\metre}$ unter der Erde ein Tank aus $\SI{380}{\cubic\meter}$ $\ce{C2Cl4}$, in welchem aus der Umwandlungsreaktion $\nu_e + ^{37}\ce{Cl} \rightarrow ^{37}\ce{Ar} + e^-$ durch genaues Zählen der entstandenen Argon-Atome die solaren Neutrinos gemessen werden konnten.
Das Ergebnis stand im Widerspruch mit dem solaren Standardmodell, welches bis zu diesem Zeitpunkt jedoch gute Aussagen über die Eigenschaften der Sonne wie Energiespektrum, Helligkeit oder Temperatur machte.
Betrachtet man das Ergebnis über die komplette Messphase (1970 bis 1994), so ist die Anzahl der gemessenen Neutrions um einen Faktor $3$ zu gering.
Auch weitere spätere Expermente wie Kamiokande oder Gallex, jeweils mit anderen Energieschwellen für die Neutrinos, konnten diese Abweichungen bestätigten.

Beim atmosphärischen Neutrinoproblem werden die Neutrinos betrachtet, welche bei Reaktionen von astrophysikalischen Teilchen mit der Atmosphäre entstehen.
Hierbei wurde das beobachtete Verhältnis von $\nu_\mu$ zu $\nu_e$ im Vergleich zu theoretischen Erwartungen von vielen Experimenten als wesentlich geringer gemessen.

Beide Probleme können letztendlich durch die Einführung von Neutrinooszillationen gelöst werden:
Hierbei können Neutrinos einer bestimmten Familie während der Propagation ihre Familie wechseln, wodurch sich die beobachteten Verhältnisse verändern.
Die erste Theorie von Neutrinooszillationen geht auf Bruno Pontecorvo im Jahr 1957 zurück.

\subsection{Beobachtung von Neutrinooszillationen}
Zur Beobachtung der Neutrinooszillationen wurden weitere Experimente durchgeführt.
So wurden mit dem Sudbury Neutrino Observatory, einem Detektor bestehend aus einem großen $\ce{D2O}$-Tank sowie Photomultipliern, von 1999 bis 2006 Daten aufgenommen.
Elektronneutrinos können bei diesem Experiment über geladene Ströme das Neutron im Deuterium zu einem Proton umwandeln, wobei sich das Neutrino in ein Elektron umwandelt.
Dieses Elektron kann widerum über sein Cherenkovlicht im Wasser nachgewiesen werden.
Während diese Rekation nur durch Elektronneutrinos ausgelöst werden kann, können Neutrinos aller Familien über neutrale Ströme ($d + \nu \rightarrow n + p + \nu$, Disintegration) sowie elastische Elektronstreuuung ($\nu + e^- \rightarrow \nu + e^-$) wechselwirken und somit nachgewiesen werden.
Somit konnten die fehlenden Elektronneutrinos aus dem solaren Neutrinoproblem als Neutrinos anderer Generationen gemessen werden.
Auch die Ergebnisse von Super-Kamiokande, einem Experiment mit ähnlichem Aufbau, sind mit der Theorie von Neutrinooszillationen gut kompatibel.

Für diese Beobachtungen erhielten Takaaki Kajita (Super-Kamiokande) und Arthur McDonald (Sudbury Neutrino Observatory) 2015 für die Entdeckung von Neutrinooszillationen.

\subsection{Theorie der Neutrinooszilationen und weitere Messungen}
Ähnlich wie bei der CKM-Matrix für die Quarkgenerationen exisitert die PMNS-Matrix, welche die Masseneigenzustände an die Eigenzustände der schwachen Wechselwirkung koppelt.
Sie besitzt 3 freie Parameter sowie eine Phase, dessen Bestimmung bzw. Messung wichtig für das genaue Verständnis der Neutrinooszilationen ist.
Relevant dabei ist insbesondere, dass die Oszillationswahrscheinlichkeit nicht nur von der zurückgelegten Strecke, sondern auch von den Massendifferenzen der Neutrinos abhängig ist:
Ohne Neutrinomassen können keine Neutrinooszillationen existieren, so dass der Nachweis von Oszillationen direkt aussagt, dass Neutrinos eine von null verschiedene Masse besitzen.
Dies entspricht bereits einer Erweiterung des Standardmodells der Teilchenphysik.

Experimente wie das T2K-Experiment sollen die Parameter der PMNS-Matrix bestimmen.
Bei diesem Experiment wird ein mittels Beschleuniger erzeugter Neutrinobeam verwendet, und mittels eines nahen sowie eines fernen Detektors (Super-Kamiokande) werden die während der propagierten Strecke auftretenden Oszillationen untersucht.

Ein weiterer Effekt in Kombination mit Neutrinooszillationen ist der Mikheyev-Smirnov-Wolfenstein-Effekt: 
Er besagt, dass sich das Oszillationsverhalten in Materie von dem im Vakkum unterscheidet.
Diese Asymmetrie konnte durch Experimente wie Kamland und Super-Kamiokande mittels einer tageszeitabhängigen Messung bestimmt werden:
Je nachdem, wie viel Erdmaterial zwischen der Sonne und dem Detektor liegt (was tageszeitabhängig ist), verhalten sich die Oszillationen unterschiedlich.

Relevant ist dieser Effekt ebenfalls für das Problem der sogenannten Massenhierarchie der Neutrinos:
Die Beträge der Massendifferenzen der Neutrinos können über Messobservablen bestimmt werden.
Das Vorzeichen kann jedoch über den MSW-Effekt lediglich für einen Abstand bestimmt werden, d.h. für $\Delta m_{21}$. Ob sich die Masse des dritten Neutrinos relativ oberhalb oder unterhalb dieser beiden Massen befindet, ist, genauso wie die absoluten Massen, unbekannt.

In der Neutrinophysik ist somit insbesondere die Messung der Matrixelemente Thema der aktuellen Forschung.
Zudem sollen die Massendifferenzen genauer bestimmt werden.
Auch neue Eigenschaften, wie die Existenz von sterilen Neutrinos oder Neutrinos als Majoranateilchen, werden bis heute untersucht.

\subsection{Diskussion}

In der Diskussion zum Vortrag wurden insbesondere Ergebnisse sowie experimentelle Besonsderheiten besprochen.
Zudem kam die Frage auf, wieso in der Atmosphäre mehr $\mu^+$ als $\mu^-$ Teilchen auftreten.
Diese Frage lässt sich dadurch beantworten, dass insbesondere Protonen als kosmische Strahlung auf die Atmosphäre treffen und somit aufgrund der Ladungserhaltung insgesamt mehr positiv als negativ geladene Teilchen auftreten.
Anschließend an diese Frage wurde besprochen, wieso das Verhältnis von $\mu^+$ zu $\mu^-$ zu höheren Energien ansteigt.
Dies liegt daran, dass die Teilchen höherer Energien insbesondere als Primärteilchen aus der Reaktion mit der Atmosphäre auftreten, und somit vornehmlich positiv geladen sind.
\newpage


\section{Die W-/Z-Boson Entdeckung}

\chapterauthor{Jean-Marco Alameddine, 07.12.2018}

\subsection{Historische Entwicklung der schwachen Wechselwirkung}

Die Geschichte der theoretischen Beschreibung der schwachen Wechselwirkung geht zurück auf Fermi im Jahre 1933.
Er beschrieb den $\beta^-$-Zerfall über eine direkte Vier-Fermionen Wechselwirkung ohne Austauschteilchen.
Während diese Theorie für kleine Energien (verglichen mit der Masse der heute bekannten Eichbosonen W und Z) eine gute Näherung beschreibt, treten für höhere Energien theoretische Probleme und experimentelle Diskrepanzen in Fermis Theorie auf.
Trotzdem stellte diese Theorie der schwachen Wechselwirkung über viele Jahre hinweg eine gute Vorhersage dar, die auf viele Prozesse angewendet sowie an neue Phänomene wie die Paritätsverletzung angepasst werden konnte.

Um Probleme wie die Divergenzen in höheren Ordnungen bzw. bei hohen Energien zu beheben beschrieben Glashow, Weinberg und Salam im Jahre 1968 die elektroschwache Wechselwirkung.
Es handelt sich dabei um eine vereinheitlichte Theorie der Quantenelektrodynamik und der schwachen Wechselwirkung.
Vorhersagen die durch diese Theorie getroffen werden sind insbesondere das masselose Photon sowie die massebehafteten Austauschteilchen $\text{W}^+$, $\text{W}^-$ und $\text{Z}^0$.
Aufgrund von Letzterem werden innerhalb der elektroschwachen Theorie sogenannte neutrale Ströme vorhergesagt, bei denen ein Neutrino mit einem leptonischen oder hadronischen Teilchen unter Austausch eines $\text{Z}^0$-Bosons wechselwirkt.

Dieses Phänomen konnte im Juli 1973 mithilfe des Gargamelle Detektors, einer Blasenkammer am CERN, nachgewiesen werden.
Hierbei wird ein fokussierter Neutrinostrahl auf eine mit Freon gefüllte Kammer gelenkt. 
Mithilfe von Lichtblitzen und Kameras werden ionisierende Teilchen, die Gasblasen als Spur hinterlassen, nachgewiesen.
Hierbei konnte die eindeutige Signatur von neutralen Strömen nachgewiesen werden, welche durch die elektroschwache Theorie vorhergesagt wurden.
Zudem konnten die Massen der in dieser Theorie enthaltenen Bosonen zu 
\begin{align*}
	M_\text{W} &\approx \left(60-80\right)\si{\giga\electronvolt}\\
	M_\text{Z} &\approx \left(75-92\right)\si{\giga\electronvolt}
\end{align*}
abgeschätzt werden.
Zum Zeitpunkt der Entdeckung existiere jedoch kein Experiment mit ausreichender Schwerpunktsenergie $\sqrt{s}$ um diese Teilchen tatsächlich zu erzeugen und somit eindeutig nachzuweisen.

\subsection{Beschleuniger zur Entdeckung der W- und Z-Bosonen}
Seit Juni 1976 war am CERN das Super Proton Synchrotron (SPS) im Betrieb.
Es handelte sich dabei um ein Fixed-Target Experiment für Protonen, wobei diese auf eine Strahlenergie von $E = \SI{400}{\giga\electronvolt}$ beschleunigt wurden.
Als Schwerpunktsenergie steht hierbei jedoch nur eine Energie von $\sqrt{s} = \sqrt{2 E m} \approx \SI{27}{\giga\electronvolt}$ zur Verfügung, was nicht zum Nachweis des W- oder Z-Bosons ausreicht.

Eine mögliche Idee, um höhere Schwerpunktsenergien zu erreichen, stellen Collding-Beam-Experimente dar.
Für Proton-Proton Kollisionen beträgt die Schwerpunktsenergie, unter Berücksichtigung der Partonstruktur der Hadronen, $\sqrt{s} = 2 \cdot E \cdot x$ mit $x \approx \num{0.2}$.
Eine Weiterentwicklung dieser Idee von Carlo Rubbia und Peter McIntyre war es, anstatt Proton-Proton Kollisionen, Protonen mit Antiprotonen kollidieren zu lassen.
Hierzu wäre es ausreichend, ein bestehendes Proton-Synchrotron umzubauen, so dass Protonen und Antiprotonen entgegengesetzt in derselben Vakuumröhre beschleunigt werden können.
Auf Grundlage dieser Idee wurde im Januar 1978 begonnen, das SPS in einen Proton-Antiproton-Collider, das Super Proton Antiproton Synchrotron (Sp$\overline{\text{p}}$S) umzubauen.
Hierbei konnten bei einer Strahlenergie von $E = \SI{300}{\giga\electronvolt}$ Schwerpunktsenergien von etwa $\sqrt{s} = \SI{120}{\giga\electronvolt}$ erreicht werden, was für die vorhergesagte Erzeugung der massebehafteten Eichbosonen der schwachen Wechselwirkung ausreichend ist.

Das Sp$\overline{\text{p}}$S wurde von Juli 1981 bis 1991 als Speicherring für Protonen und Antiprotonen betrieben.
Hierzu war es insbesondere nötig, das Vakuum in der Röhre deutlich zu verbessern sowie die entgegengesetzte Injektion und Beschleunigung in dieser Röhre zu ermöglichen.
Um eine ausreichende Anzahl von Antiprotonen für den Speicherring zu erzeugen wurde der Antiproton Accumulator (AA) verwendet.
Die Antiprotonen werden erzeugt, indem ein Protonenstrahl vom Proton Synchrotron (PS) auf ein Target geleitet wird.
Die entstehenden Antiprotonen werden mit einem Spektrometer ausgewählt und in den AA geleitet.
Um die Antiprotonen für das Sp$\overline{\text{p}}$S verwenden zu können muss insbesondere der Phasenraum der Antiprotonen verkleinert werden.
Hierzu wird das Prinzip der stochastischen Kühlung verwendet, welches von Simon van der Meer 1972 entwickelt wurde:
Teilchen bzw. Teilchengruppen, welche nicht der Idealbahn im Speicherring folgen, führen Oszillationen durch.
Um diese zu unterdrücken werden die Abweichungen mithilfe eines "Pick-up" gemessen und ein Signal ausgegeben.
Dieses wird vom einem "Kicker" aufgenommen und eine elektromagnetische Korrektur auf den Teilchenstrahl angewendet.
Hierdurch konnte der Phasenraum um einen Faktor $\num{1e9}$ reduziert werden.

\subsection{Detektoren zur Entdeckung der W- und Z-Bosonen}

Zum Nachweis der Signaturen der W- und Z-Bosonen wurden zwei Experimente am Sp$\overline{\text{p}}$S errichtet:
Das Experiment UA1 war ein großer, multifunktionaler $4\pi$-Detektor mit Driftkammer, Kalorimetern und Myonkammern.
Im Gegensatz dazu war das zweite Experiment, UA2, spezialisiert auf den Nachweis von Elektronen unter Nutzung von präzisen, hochaufgelösten Kalorimetern.

Das nachzuweisende W-Boson zerfällt primär in ein Quark-Antiquark-Paar, welches zwei Jets bildet.
Da diese Signatur jedoch stark überlagert ist durch Untergrund aus harter Streuung der Partonen wird das W-Boson durch die Signatur $\text{W} \rightarrow l \nu_l$ nachgewiesen.
Hierbei tritt ein isoliertes Elektron mit hohem transversalen Impuls auf sowie ein Neutrino, welches über einen fehlenden transversalen Impuls nachgewiesen werden kann.
Die von UA1 beobachteten Events mit einem Elektron mit hohem transversalen Impuls weisen, wie in Abbildung \ref{fig:boson} dargestellt, in allen Fällen einen antiparallel dazu liegenden fehlenden transversalen Impuls auf, der dem Neutrino zugeordnet werden kann.
Hierdurch konnte im Januar 1983 die Entdeckung des W-Bosons bekannt gegeben werden.

\begin{figure}
  \centering
  \includegraphics[height=10.0cm]{ressources/Screenshot_2018-12-04_18-22-25.png}
  \caption{Ergebnisse von UA1 aus dem Jahre 1982 \cite{boson}}
  \label{fig:boson}
\end{figure}

Die nachzuweisende Reaktion des W-Bosons ist $\text{Z} \rightarrow \text{e}^+ \text{e}^-$.
Als Signatur existieren dabei zwei Kalorimetereinträge, auf die jeweils eine isolierte Spur eines Elektrons zeigt.
An UA1 konnten vier solcher Ereignisse sowie ein Myon-Antimyon-Paar nachgewiesen werden, welche mit einer invarianten Masse von ca. $M_\text{Z} \approx \SI{95.2}{\giga\electronvolt}$ kompatibel sind.
Zusammen mit drei von UA2 gemessenen Ereignissen konnte im Juni 1983 die Entdeckung des Z-Bosons bekannt gegeben werden.

\subsection{Weitere Entwicklungen}

Für ihren Beitrag zur Entdeckung der W- und Z-Bosonen erhielten Carlo Rubbia und Simon van der Meer 1984 den Nobelpreis in Physik.

In den folgenden Jahrzehnten folgten Präzisionsmessungen der Massen $M_\text{Z}$ und $M_\text{W}$, die insbesondere am Tevatron sowie am Large-Electron-Positron Collider durchgeführt wurden.
\newpage


\section{CP-Verletzung im Kaon-Sektor}

\label{sec:cpv}

\chapterauthor{Johannes Kollek, 14.12.2018}

\subsection{Motivation}

Die CP-Verletzung bescheibt die Tatsache, dass ein System unter Spiegelung aller Raumkoordinaten (P für Parität) sowie gleichzeitiger Vertauschung aller Teilchen mit ihren Anteiteilchen (C für charge) physikalisch nicht invariant ist.
Während die P-Verletzung bereits durch das Wu-Experiment in der schwachen Wechselwirkung gezeigt wurde, ist dieses Experiment mit CP-Invarianz kompatibel.

Die Beobachtung von CP-Verletzung ist insbesondere bei der Erklärung der Baryonasymmetrie von Bedeutung: Die Sacharowkriterien geben die Beobachtung von CP-Verletzung als eines von drei notwendigen Kriterien zur Erklärung der Baryogenese vor.

\subsection{Das Kaonsystem}

Kaonen, d.h. Mesonen mit einem Strange Quark sowie einem zusätzlichen leichten Quark, wurden erstmals 1947 in der kosmischen Höhenstrahlung entdeckt.
Um zwischen neutralen Kaonen als Kombination von $\overline{\text{s}}\text{d}$ und $\overline{\text{d}}\text{s}$ unterscheiden zu können, existieren die von Gell-Mann und Nishijima angegebenen Teilchen $K^0$ und $\overline{K}^0$. 
Im Jahre 1955 wurden von Gell-Mann und Pais die Linearkombinationen
\begin{align*}
	\ket{K_1} &= \frac{1}{\sqrt{2}} \left( \ket{K^0} + \ket{\overline{K}^0} \right) \rightarrow \num{2} \pi \\
	\ket{K_2} &= \frac{1}{\sqrt{2}} \left( \ket{K^0} - \ket{\overline{K}^0} \right) \rightarrow \num{3} \pi
\end{align*}
vorgeschlagen, wobei beide Zerfälle CP-Invariant sind.
Aufgrund der unterschiedlichen Phasenräume unterscheiden sich die Lebensdauern der beiden Zerfälle deutlich, so dass ein langlebiges und ein kurzlebiges Kaon exisitiert.

\subsection{Experimentelle Vermessung des Kaon-Systems}

Das Adair-Experiment beschäftigte sich mit der Messung der Regeneration eines $K_2$-Strahles.
Prinzipiell wechselwirken $K^0$ und $\overline{K}^0$ unterschiedlich mit Materie, so dass ein zuvor reiner Strahl aus $K_2$ Teilchen nach einer gewissen Propagation auch $K_1$ Anteile besitzt. 
Die vom Experiment gemessene Regenerationsrate war jedoch größer als erwartet.

Um diese Ergebnisse zu überprüfen wurde 1963 von Fitch und Cronin ein Experiment am Alternating Gradient Synchrotron (AGS) in Brookhaven vorgeschlagen.
Hierzu wird ein zweiarmiges Spektrometer mit Funkenkammern vor und hinter einem Magneten verwendet.
Um Interaktionen gering zu halten wird der Bereich hinter dem Kollimator mit Heliumgas gefüllt.
Das Messprinzip ist in Abbildung \ref{fig:kaon} dargestellt:
Während die geladenen Pionen rekonstruiert werden können und deren deoponierte Energie und Winkel gegen die Strahlachse bestimmt werden können, lassen sich die neutralen Kaonen nur über ihren fehlenden Beitrag rekonstruieren.
Somit kann aus dem summierten Winkel, unter dem die gemessenen Impulse auftreten, darauf geschlossen weden, ob es sich um einen $2\pi$-oder $3\pi$-Zerfall handelt.
\begin{figure}
  \centering
  \includegraphics[height=6.0cm]{ressources/kaon.png}
  \caption{Messprinzip zur Unterscheidung von $2\pi$ und $3\pi$ Zerfällen \cite{kaon}.}
  \label{fig:kaon}
\end{figure}

Nachdem andere Effekte als Ursache ausgeschlossen werden konnten, bestätigte das Experiment 1964, dass es Zerfälle von $K_2^0$ Teilchen in $2\pi$ gibt.
Die vom Experiment beobachteten Teilchen entsprechen somit den Linearkombinationen
\begin{align*}
	\ket{K_\text{S}} &= \frac{1}{\sqrt{1+|\epsilon|^2}} \left( \ket{K_1} - \epsilon \ket{K_2} \right) \rightarrow \num{2} \pi  \\
	\ket{K_\text{L}} &= \frac{1}{\sqrt{1+|\epsilon|^2}} \left( \ket{K_2} + \epsilon \ket{K_1} \right) \rightarrow \num{3} \pi 
\end{align*}
wobei die Zerfälle für $\epsilon \neq 0$ CP-verletzend sind.
Der Parameter $\epsilon$ entspricht somit der CP-Verletzung in der Mischung.

%https://de.wikipedia.org/wiki/Kaon#Entdeckung

\subsection{CP-Verletzung in der Mischung und direkte CP-Verletzung}

Um die zeitliche Entwicklung des Kaonsystems zu beschreiben wird die Schrödingergleichung für $\ket{K^0(t)}$ und $\ket{\overline{K}^0(t)}$ aufgestellt und gelöst.
Hierbei führen Box-Diagramme höherer Ordnung zu Nebendiagonalelementen im Hamiltonian.
Durch das Diagonalisieren des Hamilotnian erhält man die bekannten Zustände $\ket{K_\text{S}}$ und $\ket{K_\text{L}}$.

Die CP-Verletzung in der Mischung tritt nun auf, wenn sich die Übergangswahrscheinlichkeiten von $K^0$ und $\overline{K}^0$ unterscheiden, d.h. 
\begin{align*}
	P\left( K^0 \rightarrow \overline{K}^0 \right) \neq P\left( \overline{K}^0 \rightarrow K^0 \right)
\end{align*}
erfüllt ist.

Zusätzlich kann auch eine direkte CP-Verletzung beobachtet werden, wenn 
\begin{align*}
	P\left( A \rightarrow B \right) \neq P\left( \overline{A} \rightarrow \overline{B} \right)
\end{align*}
für Kaonen erfüllt ist.
Während der Parameter für die CP-Verletzung in der Mischung $\epsilon$ heißt, existiert auch der Parameter $\epsilon^{\prime}$ welcher die direkte CP-Verletzung beschreibt.
In guter Näherung können diese Parameter über die Amplitudenverhältnisse von $K_\text{L}$- und $K_\text{S}$-Zerfällen in geladene bzw. ungeladene Pionen beschrieben werden.

Erste Messungen zur Suche nach direkter CP-Verletzung, welche deutlich seltener als indirekte CP-Verletzung in der Mischung zu messen ist, fanden 1988 am Experiment Na31 am CERN sowie N731 am Fermilab statt.
Aufgrund von fehlender Präzision lieferten diese Experiente noch keine signifikanten Ergebnisse.
Eine finale Bestätigung von direkter CP-Verletzung lieferte erst das Experiment Na48 am CERN.
Dieses Experiment konnte eine simultane Messung von $K_\text{L}$ und $K_\text{S}$ Zerfällen durchführen, insbesondere durch die Nutzung von Driftkammern sowie Kalorimetern zum Nachweis der Photonen im Zerfall neutraler Pionen.

\subsection{Ausblick}

Bis heute existieren Fragen im Bereich der CP-Verletzung.
Einerseits kann neben dem Kaon-System auch der B-Sektor und D-Sektor untersucht werden.
Andererseits ist eine CP-Verletzung innerhalb der starken Wechselwirkung in der Theorie erlaubt, konnte bisher jedoch noch nicht nachgewiesen werden.
Eine mögliche Lösung dieses sogenannten starken CP-Problems schlägt eine neue Symmetrie vor, aus der durch Symmetriebrechung das sogenannte Axion entsteht.
Dieses Teilchen, welches trotz intensiver Suchen noch nicht beobachtet wurde, wäre zusätzlich ein Kandidat für dunkle Materie.

\newpage


\section{Die Entdeckung des kosmischen Mikrowellenhintergrundes}

\chapterauthor{Yvonne Kasper, 21.12.2018}

\subsection{Theorien zur Entstehung des Universums}

Vor der Entdeckung der komischen Hintergrundstrahlung existierten insbesondere zwei verschiedene Theorien zur Entstehung des Universums.

Die erste Theorie, die sogenannte Steady State Theorie, basiert auf dem kosmologischen Prinzip.
Sie wurde insbesondere von Vertretern wie beispielsweise Albert Einstein unterstützt und geht von einer Expansion des Universums mit einer langsamen, aber kontinuierlichen Erzeugung von Materie aus. 
Hierdurch bliebe die Dichte im Universum erhalten.
Im Rahmen dieser Theorie existiert jedoch keine Erklärung für eine kosmische Hintergrundstrahlung, weshalb sie nach dessen Entdeckung 1964 von den meisten Forschern verworfen wurde.

Auf der anderen Seite existierte die Urknalltheorie, welche beispielsweise von Alexander Friedmann, Ralph Alpher oder George Gamow vertreten wurde.
Letzterer beschäftigte sich mit der Frage nach dem Ursprung der Elemente im Universum.
Hierbei sind insbesondere der Anteil von \SI{75}{\percent} Wasserstoff und \SI{24}{\percent} Helium in der interstellaren Materie zu nennen.
Nach Gamows Theorie war das Universum zu einem frühen Zeitpunkt eine heiße "Suppe" aus Protonen, Neutronen, Elektronen sowie thermaler Strahlung.
Durch Fusionsprozesse, für die hohe Temperaturen im frühen Universum von bis zu \SI{e9}{\kelvin} benötigt wurden, entsteht zunächst Deuterium aus Protonen und Neutronen und im weiteren Verlauf Helium.
Durch das Hinzufügen weiterer Protonen und Neutronen würden weitere, schwerere Elemente erzeugt werden.
Damit im Rahmen der Theorie nicht alle Protonen und Neutronen zu schwereren Kernen verschmelzen muss sich das Universum ausdehnen damit die angegebenen Fusionsprozesse durch das Ausdünnen der Materie stoppen.
Der im Rahmen dieser Theorie beschrieben "Feuerball" würde durch die Ausdehnung des Universums auf bis zu \SI{5}{\kelvin} abkühlen.
Diese Vorhersage einer kosmischen Hintergrundstrahlung wurde in einem weitgehend unbeachteten Paper von Alpher, Bethe und Gamow bereits 1948 veröffentlicht. 

\subsection{Die experimentelle Entdeckung der kosmischen Hintergrundstrahlung}

Die Physiker Robert Woodrow Wilson und Arno Allen Penzias arbeiteten ab 1963 bzw. 1962 bei den Bell Laboratories.
Zuvor beschäftigten sich beide im Rahmen ihrer Doktorarbeiten bereits mit experimentellen astrophysikalischen Themen.
Ihr neuer Arbeitgeber, die Bell Laboratories, waren ein Forschungs- und Entwicklungszentrum für die Bell Telephone Company, die US Regierung sowie die Grundlagenforschung.

Wilson und Penzias bauten eine Horn-Reflektorantenne, siehe Abbildung \ref{fig:cmb}, mit dem ursprünglichen Ziel Experimente im Bereich der Radioastronomie und Satellitenkommunikation durchzuführen. 
Ihre Antenne mit einer Länge von 15 Metern und einer \SI{6}{\metre} mal \SI{6}{\metre} großen Öffnung eignete sich gut für die Messung von schwachen Signalen.

\begin{figure}
  \centering
  \includegraphics[height=8.0cm]{ressources/cmb.jpg}
  \caption{Von Wilson und Penzias gebaute Hornantenne mit welcher der kosmische Mikrowellenhintergrund unbeabsichtigt entdeckt wurde \cite{cmb}}
  \label{fig:cmb}
\end{figure}

Bereits erste Messungen zeigten einen unvorhergesehenen Untergrund von \SI{3.5}{\kelvin} welcher trotz mehrmonatigen Arbeiten am Empfängersystem nicht beseitigt werden konnte.
Mögliche Störungsquellen, die hierbei berücksichtigt wurden, waren beispielsweise Atmosphärenstrahlung, Störsignale der nahegelegenen Großstadt New York City, von der Regierung durchgeführte Atombombentest jedoch auch Hinterlassenschaften von Tauben in der Antenne.

Zur Interpretation der Ergebnisse sprach Penzias mit Henry Dicke aus Princton, welcher mit seiner Arbeitsgruppe bereits vorher ein ähnliches Experiment zur Untersuchung einer möglichen kosmischen Hintergrundstrahlung vorbereitet hatte.
Es wurde daraufhin von Wilson und Penzias ein Paper mit ihren experimentellen Ergebnissen sowie zeitgleich ein Paper von Dicke zur Interpretation ebendieser Ergebnisse publiziert.
Hierbei wurde der Untergrund korrekterweise als kosmische Hintergrundstrahlung interpretiert.
1978 erhielten Penzias und Wilson für ihre experimentelle Entdeckung den Nobelpreis.

\subsection{Folgende Untersuchungen der kosmischen Hintergrundstrahlung}

Nachfolgende Experimente konnten eine homogene, isotrope und unpolarisierte Hintergrundstrahlung bei ca. \SI{3.5}{\kelvin} bestätigen.
Hierzu gehörten neben erdgebundenen Experimenten auch satellitengestützte Messungen wie COBE, (W)MAP und Planck sowie Ballonexperimente wie BOOMERanG und MAXIMA.

Das COBE (Cosmic Background Explorer) Experiment untersuchte beispielsweise von 1989 bis 1993 die Infrarotstrahlung sowie hohe Frequenzen des Spektrums.
Hierbei konnte bestätigt werden, dass die kosmische Hintergrundstrahlung ein perfektes Schwarzkörperspektrum beschreibt.
Zudem wurden mit dem "Differential Microwave Radiometer" von COBE Anisotropien im CMB untersucht. 
Diese Anisotropien wurden bereits 1967 durch theoretische Berechnungen von Sachs und Wolf vorhergesagt.
Im Rahmen dieses Experimentes konnte die Temperatur des kosmischen Mikrowellenhintergrundes zu $T = \SI{2.728}{\kelvin}$ mit Temperaturfluktuationen, d.h. Anisotropien, in der Größenordnung von \num{e-5} gemessen werden.
George Smoot und John Mather erhielten für die Entdeckung dieser Anisotropien sowie dem Nachweis des Schwarzkörperspektrums 2006 den Nobelpreis.

Ein weiteres Experiment, genannt BOOMERanG (Balloon Oberservations Of Millimetric Extragalactic Radiation and Geophysics) wurde von 1998 bis 2003 in der Antarktis betrieben und konnte Messungen in bis zu \SI{42}{\kilo\metre} Höhe durchführen.
Hierbei konnten die Anisotropien in der Hintergrundstrahlung mit noch höherer Genauigkeit vermessen werden und aus diesen Ergebnissen auf die flache Geometrie des Universums geschlossen werden.

Neben dem komischen Mikrowellenhintergrund wird theoretisch auch ein Neutrinohintergrund vorhergesagt. Aufgrund der frühen Entkopplung der Neutrinos von den Elektronen und Photonen im Universum sowie dem "Reheating" des CMB, wobei durch Entropieübertrag auf die Photonen die Temperatur des Mikrowellenhintergrundes im Vergleich zum Neutrinohintergrund größer geworden ist, wird der Neutrinohintergrund bei einer Temperatur von ca. \SI{1.9}{\kelvin} vorhergesagt.
Die Messung einer solch geringen Temperatur bei Neutrios ist jedoch experimentell schwierig und war deshalb noch nicht erfolgreich.
\newpage


\section{Die Entdeckung des Higgs-Bosons}

\chapterauthor{Patrick Schmidt, 11.01.2019}

\subsection{Geschichte und Theorie der spontanen Symmetriebrechung}
Die in den 1960er Jahren durch Glashow, Weinberg und Salam beschriebene elektroschwache Theorie zur vereinheitlichten Erklärung der elektromagnetischen und schwachen Wechselwirkung besaß bis ins Jahr 1964 keine Erklärung des Zustandekommens der Massen der dazugehörigen Eichbosonen.
Erst der durch Peter Higgs eingeführte Higgs-Mechanismus lieferte durch die Einführung eines skalaren Feldes eine mit Massen kompatible Theorie.

Grundlegend für den Higgs-Mechanismus ist das Phänomen der spontanen Symmetriebrechung, welches auch beispielsweise in der Festkörperphysik bei einem Ferromagneten beobachtet werden kann.
Ein einfaches Beispiel ist durch das in Abbildung \ref{fig:higgs} angegebene Potential gegeben.
Hier ist das Minimum des Potentials vom Ursprung verschoben.
Gleichzeitig existieren mehrere, prinzipiell gleichberechtigte Minima, welche alle denselben Vakuumserwartungswert besitzen.
Die Wahl eines dieser Minima entspricht dabei der spontanen Symmetriebrechung, analog zur Wahl einer Vorzugsrichtung bei Abkühlung eines Ferromagnetens unter die Curie-Temperatur.
Der Lagrangian, welcher das oben beschriebene Potential verkörpert, kann nun um den neuen Vakuumerwartungswert betrachtet werden.
Hierbei tritt, wie durch das Goldstone-Theorem beschrieben, ein masseloses Goldstone-Boson auf.

\begin{figure}
  \centering
  \includegraphics[height=5.0cm]{ressources/higgspotential.png}
  \caption{"Mexiko-Hut-Potential" als Beispiel zur spontanen Symmetriebrechung \cite{Ellis:2012465}}
  \label{fig:higgs}
\end{figure}

Im Rahmen des Higgs-Mechanismus treten Wechselwirkungen zwischen deim Higgs-Feld sowie den Feldern der Eichbosonen auf.
Durch diese Kopplungsterme treten sowohl die massebehafteten Eichbosonen auf als auch deren Kopplung an das Higgs-Boson.
Die im Standardmodell gebrochene Symmetrie ist dabei die $SU(3) \times U(1)$-Symmetrie.

\subsection{Suche nach Singaturen des Higgs-Bosons}
Während der Higgs-Mechanismus zwar eine theoretische Erklärung der Massen der Eichbosonen liefert, würde nur ein tatsächlicher Nachweis des Higgs-Bosons dazu führen, dass ebendieser Mechanismus als korrekte Theorie bestätigt werden kann.
Bei der Betrachtung der infrage kommenden Erzeugungs- und Zerfallskanäle ist zu beachten, dass die auftretenden Kopplungen an das Higgs-Boson proportional zu den beteiligten Massen sind.
Ein möglicher Prozess zur Erzeugung sowie Zerfall unter Kopplung an das schwerste bekannte Quark, das Top-Quark, ist in Abbildung \ref{fig:higgs2} dargestellt.
Zu beachten ist, dass für verschiedene mögliche Massen des Higgs-Bosons verschiedene Zerfallskanäle dominieren können.
Bei späteren Analysen ist jedoch nicht immer die einzige Betrachtung des dominiernsten Zerfallskanals sinnvoll, da häufig andere Untergrundprozesse (beispielsweise beim $\text{b}\overline{\text{b}})-Kanal$ andere hadronische Prozesse) dominieren.

\begin{figure}
  \centering
  \includegraphics[height=5.0cm]{ressources/higgsfeyn.png}
  \caption{Mögliche Erzeugung und möglicher Zerfall eines Higgs-Bosons. Die Gluonen können beispielsweise aus Protonen in pp-Kollisionen stammen \cite{higgs_production_decay}.}
  \label{fig:higgs2}
\end{figure}

\subsection{Untere Grenzen auf die Higgs-Masse durch erste experimentelle Untersuchungen}
Vor der finalen Entdeckung des Higgs-Bosons im Jahre 2012 konnten viele andere Experimente bereits Einschränkungen auf die Masse eines möglichen Higgs-Bosons treffen.

%TODO: BB_BAR Oszillationen?! 

Das Experiment NA31, welches von 1987 bis 1989 Messungen am SPS am CERN durchführte, hatte die primäte Aufgabe nach CP-Verletzung im Kaonsektor zu suchen.
Trotzdem konnte die Betrachtung der Daten unter Berücksichtigung eines möglichem Kaonzerfalls der Form
\begin{align*}
	\text{K}_\text{L}^0 \rightarrow \pi^0 \text{H}^0 \rightarrow \pi^0 e^+ e^-
\end{align*}
eine erste untere Grenze von $m_\text{H} = \SI{15}{\mega\electronvolt}$ angegeben werden.

Der LEP war ein $e^+ e^-$-Collider, welcher von 1989 bis 2000 betrieben wurde und Higgs-Bosonen über den Prozess
\begin{align*}
	e^+ e^+ \rightarrow \text{H} Z^0
\end{align*}
produzieren sollte.
Als Kanäle zur Beobachtung wurden als mögliche Endzustände 4-Jet-Zustände, Kanäle mit fehlender Energie aufgrund von zwei Neutrinos sowie leptonische Kanäle mit zwei Elektronen im Endzustand gesucht.
Zwar ergab die Suche keinen Erfolg, auf die Masse konnte jedoch ein unteres Limit von \SI{114.4}{\giga\electronvolt} gesetzt werden.

Mit den Experimenten D0 und CDF am Tevatron wurde in Daten von 2001 bis 2011 erstmals an einem $p\overline{p}$-Collider nach Higgs-Bosonen gesucht.
Hierbei kommt zur Produktion erstmals auch der in Abbildung \ref{fig:higgs2} dargestellte Kanal in Frage.
Das Experiment konnte dabei die Bereiche
\begin{align*}
	\SI{100}{\giga\electronvolt} &\leq m_\text{H} \leq \SI{108}{\giga\electronvolt} \\
	\SI{156}{\giga\electronvolt} &\leq m_\text{H} \leq \SI{177}{\giga\electronvolt} \\
\end{align*}
für die Higgs-Masse ausschließen.

\subsection{Entdeckung des Higgs-Bosons am LHC}

Studien über die Erzeugung von Higgs-Bosonen in $p\overline{p}$-Kollisionen in den 90er Jahren stellten die Grundlagen für den Bau des Large Hadron Colliders dar.
Auf Basis der durchgeführten Berechnungen wurden die Zerfallskanäle
\begin{align*}
	\text{H} &\rightarrow \gamma \gamma \\
	\text{H} &\rightarrow l^+ l^- l^+ l^- 
\end{align*}
als relevanteste Kanäle zur Detektion bei der Konstruktion des Detektoren berücksichtigt.

Nach dem ersten Betriebsbeginn im September 2008 mit einem anschließenden Betriebsstop aufgrund von nötigen Reparaturen konnte am 2010 mit Messungen bei einer Schwerpunktsenergie von $\sqrt{s} = \SI{140}{\giga\electronvolt}$ begonnen werden.
Erste Analyseergebnisse in den oben angegebenen Kanälen konnten die Masse auf einen Bereich 
\begin{align*}
	\SI{115.5}{\giga\electronvolt} &\leq m_\text{H} \leq \SI{131}{\giga\electronvolt}
\end{align*}
einschränken.


\newpage


\section{Entdeckung des Top-Quarks}

\chapterauthor{Alexander Froch, 18.01.2019}

\subsection{Übersicht der Entdeckung}

Das Top-Quark ist eines der beiden Quarks der dritten Generation und mit einer heute bekannten Masse von  $m_t = \SI{173.1(6)}{\giga\electronvolt}$ das schwerste bekannte Quark und Elementarteilchen.
Bereits im Jahre 1973 postulierten Kobayashi und Maskawa sowohl das Bottom-Quark als auch das Top-Quark theoretisch um die beobachtete CP-Verletzung beim Zerfall von Kaonen zu erklären.
Sie erhielten nach der Entdeckung des b-Quarks 1977 am Fermilab sowie des t-Quarks 1995 am Tevatron im Jahre 2008 den Nobelpreis.

Die tatsächliche Entdeckung des b-Quarks 1977 gab Anlass zur Suche nach dem t-Quark.
Im Jahre 1984 konnte durch die erfolglose direkte Suche am SLAC ein unteres Limit von \SI{23.3}{\giga\electronvolt} auf die Topmasse gegeben werden.
Die Ergebnisse des ARGUS Experiment am DESY, welches Oszillationen von B-Mesonen untersuchte, wiesen 1987 aufgrund der gemessenen Oszillationsfrequenzen, welche sensitiv auf die Topquarkmasse sind, auf eine Masse von über \SI{50}{\giga\electronvolt} hin.
Die Ergebnisse von VENUS am KEK, OPAL am LEP sowie UA2 am Sp$\overline{\text{p}}$S konnten durch weitere Suche in direkter Produktion diese Ergebnisse von ARGUS bestätigen und ebenfalls untere Limits setzen. 
Die schlussendliche Entdeckung gelang im Jahre 1995 dem Tevatron mit den Experimenten CDF und D0.

\subsection{Theorie und indirekte Messungen}

Als Produktionskanäle für das Topquark kommen insbesondere Gluon-Gluon Fusion, zwei Gluonen in einem t-Kanal sowie die Annihilation von $q\overline{q}$ infrage, wobei jeweils ein t$\overline{t}$-Paar entsteht.
Auch die Produktion von einzelnen t-Quarks ist möglich, beispielsweise über $q\overline{q}$-Annihilation oder weitere Prozesse der schwachen Wechselwirkung mit $b$-Quarks.
Die Lebenszeit von Top-Quarks ist zu gering um Baryonen zu bilden, so dass das Top primär in ein Bottom-Quark sowie ein W-Boson zerfällt.

Aus theoretischer Sicht ist das Top-Quark ebenfalls interessant zu untersuchen:
Aufgrund seiner hohen Masse koppelt es besonders stark an das Higgs-Boson und ist somit hochsensitiv auf ggf. vorhandene neue Physik, siehe auch Kapitel \ref{sec:higgs}.

Erste Hinweise auf das Vorhandensein eines Top-Quarks kann der sogenannte R-Plot geben, welcher das Verhältnis
\begin{align*}
	R = \frac{\sigma\left( e^+ e^- \rightarrow \text{Hadronen} \right)}{\sigma \left( e^+ e^- \rightarrow \mu^+ \mu^- \right)}
\end{align*}
in Abhängigkeit von der Schwerpunktsenergie angibt.
So verändert sich der Faktor $R$ mit der Schwerpunktsenergie wenn neue Quarks erzeugt werden können.
Zudem sind viele elektroschwache Variablen sensitiv auf die Topmasse so dass auch mit Messungen von diesen Variablen die Masse des Top-Quarks genauer bestimmt werden konnte.

\subsection{Entdeckung des Top-Quarks am Tevatron}

Der CDF-Detektor ist in Abbildung \ref{fig:cdf} dargestellt.
Der Detektoraufbau ist dabei insbesondere darauf ausgelegt mithilfe der Spurkammern sekundäre Vertices rekonstruieren  sowie mit den Kalorimetern Jet- und Hadronenergien bestimmen zu können.
\begin{figure}
  \centering
  \includegraphics[height=6.0cm]{ressources/cdfii_3d_p1.png}
  \caption{CDF-Detektor, mit Spurkammern (orange), elektromagnetischen Kalorimeter (rot), hadronischen Kalorimeter (blau) und Myonkammern (grün) \cite{Galtieri:2011yd}}
  \label{fig:cdf}
\end{figure}
Die Analyse wurde anhand des Kanals $t \overline{t} \rightarrow W b W \overline{W}$ durchgeführt wobei Daten bei einer Schwerpunktsenergie von $\sqrt{s} = \SI{1.8}{\tera\electronvolt}$ aufgenommen wurden.
Bei den W-Bosonen wird dabei insbesondere der Zerfall in zwei Leptonen oder ein semileptonischer Zerfall betrachtet wobei dileptonische Ereignisse in bestimmen Energiebereichen verworfen wurden um direkt erzeuge W- und Z-Bosonen ausschließen zu können.
Um die $b$-Quarks rekonstruieren zu können wird ein SVX-Tagging durchgeführt wobei ein sekundärer Vertex durch den Zerfall des b-Quarks erwartet wird.
Um Untergründe zu unterdrücken wird eine Kombination der Rekonstruktion des Zerfallsvertices mit der Identifikation der entstehenden Leptonen genutzt.
Dabei sind bei letzterer Identifikation insbesondere Untergründe von Hadronen und Elektronen aus anderen, unbekannten Quellen zu berücksichtigen.
Die Wahrscheinlichkeit, dass es sich bei den Ergebnissen nicht um statistische Fluktuationen des Untergrundes handelt konnte vom CDF zu $\num{4.8}\sigma$ bestimmt werden.
Mithilfe eines negativen Log-Likelihoodfits konnte aus den rekonstruierten Massen die Topquarkmasse zu 
\begin{align*}
	m_{t, \text{CDF}} = \left( \num{176} \pm \num{8} \pm \num{10} \right) \si{\giga\electronvolt}
\end{align*}
bestimmt werden.

Der zweite Detektor am Tevatron, das D0-Experiment, besaß neben Driftkammern zur Spurrekonstruktion einen Übergangsstrahlungsdetektor zur Identifikation von Elektronen sowie eine vorwärtsgerichtete Driftkammer um Jets zu detektieren welche ein hohes $\eta$ besitzen.
Anhand von kinematischen Anforderungen an die Leptonen und Jets in den verschiedenen Kanälen konnten passende Events ausgewählt werden und zwischen Signal und Untergrund, welcher insbesondere durch Z-Produktion sowie andere starke Prozesse entsteht, unterschieden werden.
Schlussendlich konnte die Topmasse durch Untersuchung des Kanals $t\overline{t} \rightarrow W^+ W^- b \overline{b} \rightarrow l \nu q \overline{q} b \overline{b}$ zu
\begin{align*}
	m_{t, \text{D0}} = \left( 199 \pm 20 \pm 22 \right)\si{\giga\electronvolt}
\end{align*}
rekonstruiert werden.
Diese Werte stimmen mit den Ergebnissen von CDF gut überein.

\subsection{Heutige Messungen und Ausblick der Topphysik}
\label{sec:higgs}

Mithilfe von heutigen Messungen, insbesondere am LHC, konnte der Wirkungsquerschnitt der $t\overline{t}$ Produktion sowie der Einzel-Top Wirkungsquerschnitt genauer bestimmt werden, so dass die Messungen mit der theoretischen Vorhersage übereinstimmen.
Vorherige, größere Unsicherheiten auf den Wirkungsquerschnitt waren insbesondere auf die großen Unsicherheiten auf die Luminosität am Tevatron zurückzuführen.
Auch die Masse des Topquarks wurde am LHC vom ATLAS-Experiment untersucht wobei aufgrund der guten Auflösung des vorhandenen elektromagnetischen Kalorimeters der Endzustand von Leptonen und Jets betrachtet wurde.

In aktuellen und zukünftigen Messungen wird insbesondere die Ankopplung an den Higgs-Sektor in der $t \overline{t} \text{H}$-Produktion untersucht.
Hierbei zeigen die gemessenen Wirkungsquerschnitte in den verschiedenen Kanälen, insbesondere aufgrund der fehlenden Statistik, nach wie vor große Fehler auf.
Neben den experimentellen Unsicherheiten treten jedoch auch größere theoretische Fehler, beispielsweise durch mögliche Korrekturen höherer Ordnung sowie Unsicherheiten in der Kopplungskonstante oder den Partonverteilungsfunktionen auf.
Aufgrund dieser Unsicherheiten können zum jetzigen Zeitpunkt keine Einflüsse durch neue Physik ausgeschlossen werden, möglich ist aber auch dass der Prozess theoretisch noch nicht vollständig verstanden wurde oder Monte-Carlo Simulationen noch korrigiert werden müssen.
\newpage


\section{Die Entdeckung des Strange-Quarks}

\chapterauthor{Egor Evsenin, 25.01.2019}

\subsection{Historische Einordnung}

Der Stand in der Teilchenphysik in den 40er Jahren war, dass 1933 das Positron und drei Jahre später das Myon entdeckt wurden.
Letzteres wurde zunächst jedoch als $\mu$-Meson bezeichnet und, aufgrund seiner hohen Masse, dem in der Yukawa-Theorie vorhergesagten Austauschteilchen der starken Wechselwirkung zugeordnet.
Erst mit der Entdeckung des Pions 1947 konnte diese Fehlzuordnung, welche sich auch in der wenige Jahre später gemessenen und von der Theorievorhersage abweichenden, Lebensdauer des Myons angekündigt hatte.

\subsection{Die Entdeckung eines schweren Mesons}

Bereits 1944 konnten Leprince-Ringulet und Lhéritier bei Experimenten auf einer Forschungsanlage in \SI{3613}{\metre} Höhe Hinweie auf ein schweres Meson mit eine Masse von etwa $m \approx \SI{500}{\mega\electronvolt}$ erhalten.
Eine genaue Assoziation war zu dem Zeitpunkt jedoch noch nicht möglich, insbesondere da auch das Pion zu diesem Zeitpunkt noch nicht entdeckt war.

Erst durch Untersuchungen der kosmischen Höhenstrahlung von Rochester und Butler konnte im Dezember 1947 die Entdeckung von zwei neuen "Elementarteilchen", einem geladenen und einem ungeladenem, bekanntgegeben werden.
Diese wurden, wie alle anderen neu entdeckten Mesonen, zunächst als V-Teilchen bezeichnet.
Rochester und Butler führten ihre Untersuchungen mithilfe einer Nebelkammer und einer eingebauten \SI{3}{\centi\metre} Bleiplatte, welche im Prinzip wie ein Kalorimeter genutzt werden konnte, durch. 
Im Gegensatz zu den meisten anderen Experimenten ihrer Zeit welche kosmische Strahlung betrachreten, fanden die Experimente fast auf Meereshöhe statt.

Das Prinzip der Nebelkammer, welches bereits 1911 von Charles Wilson erdacht wurde, kann die Spuren von ionisierenden Teilchen sichtbar machen.
Dabei besteht eine einfache Nebelkammer aus zwei Platten, wobei Alkohol an der oberen, heißen Platte verdampft und nach unten fällt.
Dort kühlt es sich an der unteren, kalten Platte ab so dass es dort zu einer Übersättigung des Dampfes kommt.
Ein ionisierendes Teilchen dient nun als Keim für die Kondensation in diesem übersättigtem Gemisch und mithilfe einer Kamera kann die Teilchenspur somit nachgewiesen werden.
Neben der Spurrekonstruktion können mithilfe eines Magnetfeldes sogar Impuls- und Energiemessungen der durchquerenden Teilchen durchgeführt werden.

Rochester und Butler betrieben ihre Nebelkammer insgesamt über ein Jahr und konnte die Masse der Teilchen $V^0$ und $V^+$ zu
\begin{align*}
	770 m_e < M_{V^0} < 1600 m_e \\
	980 m_e < M_{V^+} < 1200 m_e
\end{align*}
abschätzen.
Hierbei konnte die obere Massengrenze anhand der Ionisation ermittelt werden.
Zudem konnte eine Lebenszeit von $\tau \approx \SI{50e-8}{\second}$ abgeschätzt werden.

\subsection{Konsequenzen nach der Entdeckung}
 
 Nach der Entdeckung der V-Teilchen wurden insbesondere bis 1954 weitere Nebelkammerexperimente durchgeführt um die Teilchen weiter zu untersuchen.
 Hierbei hat es sich als besonders komplex herausgestellt die neutralen V-Telilchen zu untersuchen da Impuls und Winkel schwer aufzulösen waren.
 Die ermöglichte Nutzung von Kernemulsionen sowie dem Bevatron als erstes Beschleunigerexperiment ermöglichte danach genauere Untersuchungen der neuen Teilchen.
 Auf der 1953 durchgeführten Bagnères Konferenz wurden insbesondere Begriffe wie Baryon, Meson oder Hyperion definiert.
 Zudem wurde der Begriff K-Meson für schwere, strageartige Mesonen eingeführt.

 Die zugeordnete Quantenzahl der Strangeness führte auch zu der Anordnung der Mesonen bzw. Baryonen im "Eigtfold Way" durch Murray Gell-Man.
 Zeitgleich aber unabhängig von George Zweig am CERN entwickelte Gell-Man in den 60er Jahren das Quarkmodell um die entdeckten Hadronen zu strukturieren.
 Hierdurch entsteht auch die Zuordnung der entdeckten Teilchen zum Strange-Quark.

 Auch das $\tau$-$\theta$-Rätsel im Zusammenhang mit der Paritätsverletzung in der schwachen Wechselwirkung ist eng mit den Kaonen verbunden, näheres dazu in Kapitel \ref{sec:cpv}.

\subsection{Heutiges Interesse an Strange Quarks}
Um Übergänge der verschiedenen Quarkgenerationen in der schwachen Wechselwirkung zu beschreiben exisitiert die von Cabbibo, Kobayashi und Maskawa eingeführte CKM-Matrix.
Das Matrixelement $|V_{t,s}|$ ist hierbei weniger bekannt als andere Parameter der Matrix so dass deren Messung insbesondere zur Überprüfung des Standardmodells interessant ist.
Bisherige Messungen dieses Parameters konnten nur indirekt über Oszillationen im B-System durchgeführt werden.
Zukünftig interessant könnte eine direkte Messung über den Zerfall eines Topquarks in ein Strangequark sein.
Hierbei muss insbesondere der aus dem Strangquark entstehende Jet eindeutig identifiziert werden.

\newpage

\bibliographystyle{abbrv}
\bibliography{summary}

\end{document}

