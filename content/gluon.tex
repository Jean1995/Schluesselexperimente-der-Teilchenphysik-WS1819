
\section{Die Entdeckung des Gluons}

\chapterauthor{Dominik Hellmann, 09.11.2018}

\subsection{Theoretische Grundlagen der Quantenchromodynamik}
Die Quantenchromodynamik (QCD) ist die Quantenfeldtheorie der starken Wechselwirkung, d.h. der Kraft, die für das Zusammenhalten der Quarks in Hadronen verantwortlich ist.
Grundlage für die QCD ist die Yang-Mills-Theorie, welche eine im Jahre 1954 entwickelte Eichtheorie ist.
Dabei folgt die QCD aus der $SU(3)\text{c}$-Symmetriegruppe, wobei hieraus die drei Farbladungen rot, grün, blau und die dazugehörigen Antifarben resultieren.
Die Farben sind dabei analog zu den elektrischen Ladungen der QED zu verstehen.

Analog zu den Photonen der QED existieren auch Eichbosonen der QCD, welche Gluonen heißen.
Aus der Gruppentheorie folgt, dass 8 Gluonen existieren, welche den Generatoren der $SU(3)_\text{c}$ entsprechen.
Gluonen sind masselose Vektorteilchen, welche jedoch im Kontrast zur QED ebenfalls Farbladung besitzen und dementsprechend an sich selbst koppeln können.
Hieraus folgt die asymptotische Freiheit als charakteristische Eigenschaft der QCD:
Durch Vakuumpolarisationen, dem sogenannten Anti-Screening, wird die Kopplungskonstante der QCD $\alpha_{s}$ klein für große Impulsübertäge. 
Hieraus folgt, dass die einzelnen Quarks bei großen Impulsübertägen, bzw. kleinen Abständen, als asymptotisch frei verstanden werden können.
Andereresits folgt als Charakteristikum das Confinement: Bei großen Abständen steigt das Potential der QCD linear an, d.h. die Kraft für das weitere Vergrößern des Abstandes wächst linear.
Wird der Abstand weiter vergrößtert, so werden neue Teilchen in sogenannten Jets produziert, bis der Endzustand wieder farbneutral ist.
Hierdurch können farbgeladene Teilchen nur in einem farbneutralen, gebundenen Zustand beobachtet werden.
Die entstehenden Jets können in einem Detektor als Ansammlung von Teilchen beobachtet werden - Die Messung der Impulse dieser Teilchen lässt Rückschlüsse auf die Impulse der ursprünglichen Quarks zu.

\subsection{Historische Entwicklung}

Im Jahr 1964 wurden von Murray Gell-Mann erstmals Quarks als Bestandteile der Hadronen postuliert - Hierbei stellen sich die Fragen, welche Kräfte diese Elementarteilchen zusammenhalten und welche Austatuschteilchen diese Kraft vermitteln.
Weitere Hinweise auf eine mit der QCD verbundenen Quantenzahl stellten die $\Delta$-Baryonen sowie der $R$-Plot dar.
So ist die Wellenfunktion des $\Delta^{++}$-Baryons unter Vertauschung von zwei Quarks symmetrisch, falls nur die bisher bekannten Raum-, Flavor- und Spinanteile der Wellenfunktion berücksichtigt werden.
Dies steht jedoch im Widerspruch mit dem Pauliprinzip für Fermionen.
Die Einführung einer Farbe als Quantenzahl stellt eine Lösung dieses Problems dar.
Bei der Betrachtung des $R$-Plots wird das Verhältnis der Produktion von Hadronen aus $e^+ e^-$ Reaktionen in Abhängigkeit der Schwerpunktsenergie betrachtet.
Hierbei ist das Verhältnis $R$ sensitiv auf das Vorhandensein von Farbladungen und sagt experimentell drei Farbladungen vorraus.



% Vorhersage von Jets: 1975 durch Feynman und Beobachtung 1978 am DESY. 1979 wurden 3 Jet Ereignisse beobachtet.