
\section{CP-Verletzung im Kaon-Sektor}

\chapterauthor{Johannes Kollek, 14.12.2018}

\subsection{Motivation}

Die CP-Verletzung bescheibt die Tatsache, dass ein System unter Spiegelung aller Raumkoordinaten (P für Parität) sowie gleichzeitiger Vertauschung aller Teilchen mit ihren Anteiteilchen (C für charge) physikalisch nicht invariant ist.
Während die P-Verletzung bereits durch das Wu-Experiment in der schwachen Wechselwirkung gezeigt wurde, ist das Experiment mit CP-Invarianz kompatibel.

Die Beobachtung von CP-Verletzung ist insbesondere bei der erklärung der Baryonasymmetrie von Bedeutung: Die Sacharowkriterien geben die Beobachtung von CP-Verletzung als eines von drei notwendigen Kriterien zur Erklärung der Baryogenese vor.

\subsection{Das Kaonsystem}
%https://de.wikipedia.org/wiki/Kaon#Entdeckung