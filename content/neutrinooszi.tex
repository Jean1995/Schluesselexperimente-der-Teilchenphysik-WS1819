
\section{Neutrinooszillationen}

\chapterauthor{Hubertus Kaiser, 30.11.2018}

\subsection{Das solare und das atmosphärische Neutrinoproblem}

In den 1960er Jahren trat mit dem solaren Neutrinoproblem erstmals eine Beobachtung auf, welche nicht ohne Probleme mit den bisher im Standardmodell als masselos angenommenen Neutrinos gelöst werden konnte.
Heutzutage ist bekannt, dass Neutrinos in drei Generationen ($\nu_e, \nu_\mu, \nu_\tau$) auftreten.
Bereits früh war bekannt, dass die Sonne bei seiner Energiegewinnung Elektronneutrinos produziert.
Mit dem Homestakeexperiment, welches 1970 mit seiner Messung begann, wurde die Anzahl an Neutrinos aus der Sonne gemessen.
Hierzu befand sich $\SI{1.5}{\kilo\metre}$ unter der Erde ein Tank aus $\SI{380}{\cubic\meter}$ $\ce{C2Cl4}$, in welchem aus der Umwandlungsreaktion $\nu_e + ^{37}\ce{Cl} \rightarrow ^{37}\ce{Ar} + e^-$ durch genaues Zählen der entstandenen Argon-Atome die solaren Neutrinos gemessen werden konnten.
Das Ergebnis stand im Widerspruch mit dem solaren Standardmodell, welches bis zu diesem Zeitpunkt jedoch gute Aussagen über die Eigenschaften der Sonne wie Energiespektrum, Helligkeit oder Temperatur machte.
Betrachtet man das Ergebnis über die komplette Messphase (1970 bis 1994), so ist die Anzahl der gemessenen Neutrions um einen Faktor $3$ zu gering.
Auch weitere spätere Expermente wie Kamiokande oder Gallex, jeweils mit anderen Energieschwellen für die Neutrinos, konnten diese Abweichungen bestätigten.

Beim atmosphärischen Neutrinoproblem werden die Neutrinos betrachtet, welche bei Reaktionen von astrophysikalischen Teilchen mit der Atmosphäre entstehen.
Hierbei wurde das beobachtete Verhältnis von $\nu_\mu$ zu $\nu_e$ im Vergleich zu theoretischen Erwartungen von vielen Experimenten als wesentlich geringer gemessen.

Beide Probleme können letztendlich durch die Einführung von Neutrinooszillationen gelöst werden:
Hierbei können Neutrinos einer bestimmten Familie während der Propagation ihre Familie wechseln, wodurch sich die beobachteten Verhältnisse verändern.
Die erste Theorie von Neutrinooszillationen geht auf Bruno Pontecorvo im Jahr 1957 zurück.

\subsection{Beobachtung von Neutrinooszillationen}
Zur Beobachtung der Neutrinooszillationen wurden weitere Experimente durchgeführt.
So wurden mit dem Sudbury Neutrino Observatory, einem Detektor bestehend aus einem großen $\ce{D2O}$-Tank sowie Photomultipliern, von 1999 bis 2006 Daten aufgenommen.
Elektronneutrinos können bei diesem Experiment über geladene Ströme das Neutron im Deuterium zu einem Proton umwandeln, wobei sich das Neutrino in ein Elektron umwandelt.
Dieses Elektron kann widerum über sein Cherenkovlicht im Wasser nachgewiesen werden.
Während diese Rekation nur durch Elektronneutrinos ausgelöst werden kann, können Neutrinos aller Familien über neutrale Ströme ($d + \nu \rightarrow n + p + \nu$, Disintegration) sowie elastische Elektronstreuuung ($\nu + e^- \rightarrow \nu + e^-$) wechselwirken und somit nachgewiesen werden.
Somit konnten die fehlenden Elektronneutrinos aus dem solaren Neutrinoproblem als Neutrinos anderer Generationen gemessen werden.
Auch die Ergebnisse von Super-Kamiokande, einem Experiment mit ähnlichem Aufbau, sind mit der Theorie von Neutrinooszillationen gut kompatibel.

Für diese Beobachtungen erhielten Takaaki Kajita (Super-Kamiokande) und Arthur McDonald (Sudbury Neutrino Observatory) 2015 für die Entdeckung von Neutrinooszillationen.

\subsection{Theorie der Neutrinooszilationen und weitere Messungen}
Ähnlich wie bei der CKM-Matrix für die Quarkgenerationen exisitert die PMNS-Matrix, welche die Masseneigenzustände an die Eigenzustände der schwachen Wechselwirkung koppelt.
Sie besitzt 3 freie Parameter sowie eine Phase, dessen Bestimmung bzw. Messung wichtig für das genaue Verständnis der Neutrinooszilationen ist.
Relevant dabei ist insbesondere, dass die Oszillationswahrscheinlichkeit nicht nur von der zurückgelegten Strecke, sondern auch von den Massendifferenzen der Neutrinos abhängig ist:
Ohne Neutrinomassen können keine Neutrinooszillationen existieren, so dass der Nachweis von Oszillationen direkt aussagt, dass Neutrinos eine von null verschiedene Masse besitzen.
Dies entspricht bereits einer Erweiterung des Standardmodells der Teilchenphysik.

Experimente wie das T2K-Experiment sollen die Parameter der PMNS-Matrix bestimmen.
Bei diesem Experiment wird ein mittels Beschleuniger erzeugter Neutrinobeam verwendet, und mittels eines nahen sowie eines fernen Detektors (Super-Kamiokande) werden die während der propagierten Strecke auftretenden Oszillationen untersucht.

Ein weiterer Effekt in Kombination mit Neutrinooszillationen ist der Mikheyev-Smirnov-Wolfenstein-Effekt: 
Er besagt, dass sich das Oszillationsverhalten in Materie von dem im Vakkum unterscheidet.
Diese Asymmetrie konnte durch Experimente wie Kamland und Super-Kamiokande mittels einer tageszeitabhängigen Messung bestimmt werden:
Je nachdem, wie viel Erdmaterial zwischen der Sonne und dem Detektor liegt (was tageszeitabhängig ist), verhalten sich die Oszillationen unterschiedlich.

Relevant ist dieser Effekt ebenfalls für das Problem der sogenannten Massenhierarchie der Neutrinos:
Die Beträge der Massendifferenzen der Neutrinos können über Messobservablen bestimmt werden.
Das Vorzeichen kann jedoch über den MSW-Effekt lediglich für einen Abstand bestimmt werden, d.h. für $\Delta m_{21}$. Ob sich die Masse des dritten Neutrinos relativ oberhalb oder unterhalb dieser beiden Massen befindet, ist, genauso wie die absoluten Massen, unbekannt.

In der Neutrinophysik ist somit insbesondere die Messung der Matrixelemente Thema der aktuellen Forschung.
Zudem sollen die Massendifferenzen genauer bestimmt werden.
Auch neue Eigenschaften, wie die Existenz von sterilen Neutrinos oder Neutrinos als Majoranateilchen, werden bis heute untersucht.

\subsection{Diskussion}

In der Diskussion zum Vortrag wurden insbesondere Ergebnisse sowie experimentelle Besonsderheiten besprochen.
Zudem kam die Frage auf, wieso in der Atmosphäre mehr $\mu^+$ als $\mu^-$ Teilchen auftreten.
Diese Frage lässt sich dadurch beantworten, dass insbesondere Protonen als kosmische Strahlung auf die Atmosphäre treffen und somit aufgrund der Ladungserhaltung insgesamt mehr positiv als negativ geladene Teilchen auftreten.
Anschließend an diese Frage wurde besprochen, wieso das Verhältnis von $\mu^+$ zu $\mu^-$ zu höheren Energien ansteigt.
Dies liegt daran, dass die Teilchen höherer Energien insbesondere als Primärteilchen aus der Reaktion mit der Atmosphäre auftreten, und somit vornehmlich positiv geladen sind.