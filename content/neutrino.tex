

\section{Entdeckung von Neutrinos}

\chapterauthor{Felix Geyer, 23.11.2018}

\subsection{Historische Postulierung des Neutrinos}
Historisch betrachtet geht die erste Postulierung des Neutrinos auf die Betrachtung des $\beta^-$-Zerfalles um das Jahr 1930 zurück.
Zu dieser Zeit wurde der $\beta^-$-Prozess als Zweikörperzerfall $\ce{n -> p + e^-}$ betrachtet, was kinematisch zu einer diskreten Energie des Elektrons führt.
Entgegen dieser Vorhersage wurde jedoch ein kontinuierliches Energiespektrum für das Elektron gemessen.
Zusätzlich wird die Drehimpulserhaltung bei dieser Betrachtung des Zerfalles verletzt.

Dies führte zu der ersten Vorhersage eines Neutrinos durch Pauli, der ein sehr leichtes Teilchen mit halbzahligen Spin vorhersagte, welches beim $\beta^-$-Zerfall zusätzlich zum Proton und Elektron entsteht.
Bei einem Dreikörperzerfall wäre das gemessene kontinuierliche Energiespektrum des Elektrons erklärbar.
Paulis Theorie zum $\beta^-$-Zerfall wurde 1934 erstmals publiziert.

\subsection{Experimenteller Nachweis des Neutrinos}
Im Laufe der Zeit existierten mehrere verschiedene Ideen und Experimente, um einen Nachweis von Neutrinos zu ermöglichen.
Führend bei der Entwicklung dieser Ideen und Experimente waren Frederick Reines und Clyde Cowan.

Die grundlegende Idee bestand darin, Neutrinos durch den Neutroneneinfang $\overline{\nu_e} + p \rightarrow n + e^+$ nachzuweisen.
Um eine große Neutrinoquelle zu erhalten und somit den vorhandenen, großen Untergrund zu eliminieren sollte eine Atombombe gezündet werden.
Hierbei spielte auch die Tatsache eine Rolle, dass der theoretisch berechnete Wirkungsquerschnitt dieser Nachweisreaktion bereits kurz nach der Veröffentlichung von Fermis Theorie zum $\beta$-Zerfall zu $\approx \SI{1e-20}{\barn}$ berechnet wurde, d.h. die erwartete Reaktion war äußerst selten.
Als Detektor sollte ein flüssiger Szintillationsdetektor verwendet werden, welcher auch auf größeren Skalen genutzt werden kann.
Durch Paarvernichtung des Positrons aus dem $\beta^-$-Zerfall mit vorhandenen Elektronen entstehen zwei Photonen, die im Szintillationsdetektor Lichtblitze verursachen.
Diese können nun durch Photomultiplier detektiert werden.

Um auf den Aufwand der Zündung einer Atombombe verzichten zu können, und somit auch eine einfachere Reproduzierbarkeit des Experimentes zu ermöglichen, entstand eine verbesserte Idee für den experimentellen Aufbau:
Hierbei sollte auch das beim $\beta^-$-Zerfall entstehende Neutron nachgewiesen werden indem die beim Kerneinfang entstehende $\gamma$-Strahlung im Detektor nachgewiesen wird.
Dafür wurde mit Cadmium versetztes Wasser als Reaktionsmedium verwendet.
Der verzögerte Nachweis von Positron und Neutron, d.h. die Koinzidenz, wäre eine eindeutige Signatur für Neutrinos und ermöglicht eine starke Reduzierung von Untergrundereignissen.
Diese Verringerung des Untergrundes erlaubte es einen Kernreaktor als Neutrinoquelle zu verwenden.
Der Detektor wurde dabei knapp $\SI{2}{\metre}$ vom Reaktor entfernt aufgestellt.
Um die direkt vom Reaktor erzeugten Neutronen abzuschirmen wurde der Detektor in eine Borax-Wasser-Lösung getaucht.
Zudem wurde eine Anti-Koinzidenz für den Ausschluss kosmischer Strahlung sowie Quecksilber und Blei zur Abschirmung gegen natürliche Radioaktivität verwendet.
Die Messungen wurden insbesondere durch einen nach wie vor zu hohen Untergrund durch kosmische Strahlung sowie elektronisches Rauschen, vor allem durch die vielen Photomultiplier, erschwert.
Das Experiment wurde sowohl bei eingeschaltetem als auch bei ausgeschaltetem Reaktor durchgeführt, wobei auch in letzterem Fall eine Signalrate nachgewiesen wurde.
Dieser Hintergrund konnte später als kosmische Strahlung identifiziert werden.

Um eine aussagekräftige Messung zu erhalten wurde das Experiment im November 1955 an einem Reaktor im Savannah River Plant Area neu aufgebaut.
Einerseits besaß dieser Reaktor eine hohe Reaktorleistung und somit einen hohen erwarteten Antineutrinofluss, zudem befand sich der Detektor $\SI{12}{\metre}$ unter der Erde, was eine bessere Abschirmung von komischer Strahlung ermöglichte.
Als Reaktionsmedium wurde ein mit Cadmium versetzter Wassertank verwendet, welcher oben und unten von zwei Szintillatoren umgeben war.
Die Messungen wurden über 100 Tage durchgeführt, die ermittelte Signalrate betrug $\SI{3.0+-0.2}{\per\hour}$.
Um nachzuweisen, dass der erste Impuls tatsächlich durch die Positronen auftritt, wurde testweise eine Bleischicht zwischen dem Wassertank und einer der Szintillatoren aufgestellt, was zu einer geringeren Detektion von ersten Impulsen in diesem Szintillator führt.
Zur Kontrolle des durch Neutronen auftretenden zweiten Impulses wurde das Cadmium aus dem Wassertank entfernt, wodurch die Rate komplett verschwindet da kein Neutroneneinfang mehr stattfindet.
Die Resultate dieses Experimentes gelten somit als Entdeckung des Neutrinos.
Reines erhielt 1995 den Nobelpreis für seine Arbeiten im Bereich der Neutrinophysik.

\subsection{Weitere Entdeckungen und Experimente der Neutrinophysik}
Während beim Savannah River Experiment Elektronneutrinos untersucht wurden, konnte 1962 von Leon Lederman, Melvin Schwatz und Jack Steinberger am Alternating Gradient Synchrotron auch das Myonneutrino nachgewiesen werden.
Hierfür wurden Pionen mithilfe eines Fixed-Target Experimentes erzeugt.
Diese zerfielen in Myonen sowie Myonneutrinos, wobei der entstehende Teilchenstrahl von einer Stahlwand abgeschirmt wurde.
Im Gegensatz zu den Myonen können die Neutrinos diese Wand jedoch assieren und durch Kerneinfang als entstehende Myonen nachgewiesen werden.

Im Jahre 2000 wurde durch das DONUT-Experiment am Fermilab ebenfalls das Tauneutrino nachgewiesen.
Erschwerend bei dessen Nachweis ist die geringe Lebensdauer der entstehenden Tauonen beim Einfang von Neutrinos durch Neutronen.

Heutige betriebene Neutrinoexperimente sind beispielsweise ANTARES, IceCube oder (Super-)Kamiokande.
Diese Experimente dienen verschiedenen Zwecken wie dem Nachweis von atmosphärischen, solaren oder kosmischen Neutrinos der unterschiedlichen Neutrinofamilien.