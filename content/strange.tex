
\section{Die Entdeckung des Strange-Quarks}

\chapterauthor{Egor Evsenin, 25.01.2019}

\subsection{Historische Einordnung}

Der Stand in der Teilchenphysik in den 40er Jahren war, dass 1933 das Positron und drei Jahre später das Myon entdeckt wurden.
Letzteres wurde zunächst jedoch als $\mu$-Meson bezeichnet und, aufgrund seiner hohen Masse, dem in der Yukawa-Theorie vorhergesagten Austauschteilchen der starken Wechselwirkung zugeordnet.
Erst mit der Entdeckung des Pions 1947 konnte diese Fehlzuordnung, welche sich auch in der wenige Jahre später gemessenen und von der Theorievorhersage abweichenden, Lebensdauer des Myons angekündigt hatte.

\subsection{Die Entdeckung eines schweren Mesons}

Bereits 1944 konnten Leprince-Ringulet und Lhéritier bei Experimenten auf einer Forschungsanlage in \SI{3613}{\metre} Höhe Hinweie auf ein schweres Meson mit eine Masse von etwa $m \approx \SI{500}{\mega\electronvolt}$ erhalten.
Eine genaue Assoziation war zu dem Zeitpunkt jedoch noch nicht möglich, insbesondere da auch das Pion zu diesem Zeitpunkt noch nicht entdeckt war.

Erst durch Untersuchungen der kosmischen Höhenstrahlung von Rochester und Butler konnte im Dezember 1947 die Entdeckung von zwei neuen "Elementarteilchen", einem geladenen und einem ungeladenem, bekanntgegeben werden.
Diese wurden, wie alle anderen neu entdeckten Mesonen, zunächst als V-Teilchen bezeichnet.
Rochester und Butler führten ihre Untersuchungen mithilfe einer Nebelkammer und einer eingebauten \SI{3}{\centi\metre} Bleiplatte, welche im Prinzip wie ein Kalorimeter genutzt werden konnte, durch. 
Im Gegensatz zu den meisten anderen Experimenten ihrer Zeit welche kosmische Strahlung betrachreten, fanden die Experimente fast auf Meereshöhe statt.

Das Prinzip der Nebelkammer, welches bereits 1911 von Charles Wilson erdacht wurde, kann die Spuren von ionisierenden Teilchen sichtbar machen.
Dabei besteht eine einfache Nebelkammer aus zwei Platten, wobei Alkohol an der oberen, heißen Platte verdampft und nach unten fällt.
Dort kühlt es sich an der unteren, kalten Platte ab so dass es dort zu einer Übersättigung des Dampfes kommt.
Ein ionisierendes Teilchen dient nun als Keim für die Kondensation in diesem übersättigtem Gemisch und mithilfe einer Kamera kann die Teilchenspur somit nachgewiesen werden.
Neben der Spurrekonstruktion können mithilfe eines Magnetfeldes sogar Impuls- und Energiemessungen der durchquerenden Teilchen durchgeführt werden.

Rochester und Butler betrieben ihre Nebelkammer insgesamt über ein Jahr und konnte die Masse der Teilchen $V^0$ und $V^+$ zu
\begin{align*}
	770 m_e < M_{V^0} < 1600 m_e \\
	980 m_e < M_{V^+} < 1200 m_e
\end{align*}
abschätzen.
Hierbei konnte die obere Massengrenze anhand der Ionisation ermittelt werden.
Zudem konnte eine Lebenszeit von $\tau \approx \SI{50e-8}{\second}$ abgeschätzt werden.

\subsection{Konsequenzen nach der Entdeckung}
 
 Nach der Entdeckung der V-Teilchen wurden insbesondere bis 1954 weitere Nebelkammerexperimente durchgeführt um die Teilchen weiter zu untersuchen.
 Hierbei hat es sich als besonders komplex herausgestellt die neutralen V-Telilchen zu untersuchen da Impuls und Winkel schwer aufzulösen waren.
 Die ermöglichte Nutzung von Kernemulsionen sowie dem Bevatron als erstes Beschleunigerexperiment ermöglichte danach genauere Untersuchungen der neuen Teilchen.
 Auf der 1953 durchgeführten Bagnères Konferenz wurden insbesondere Begriffe wie Baryon, Meson oder Hyperion definiert.
 Zudem wurde der Begriff K-Meson für schwere, strageartige Mesonen eingeführt.

 Die zugeordnete Quantenzahl der Strangeness führte auch zu der Anordnung der Mesonen bzw. Baryonen im "Eigtfold Way" durch Murray Gell-Man.
 Zeitgleich aber unabhängig von George Zweig am CERN entwickelte Gell-Man in den 60er Jahren das Quarkmodell um die entdeckten Hadronen zu strukturieren.
 Hierdurch entsteht auch die Zuordnung der entdeckten Teilchen zum Strange-Quark.

 Auch das $\tau$-$\theta$-Rätsel im Zusammenhang mit der Paritätsverletzung in der schwachen Wechselwirkung ist eng mit den Kaonen verbunden, näheres dazu in Kapitel \ref{sec:cpv}.

\subsection{Heutiges Interesse an Strange Quarks}
Um Übergänge der verschiedenen Quarkgenerationen in der schwachen Wechselwirkung zu beschreiben exisitiert die von Cabbibo, Kobayashi und Maskawa eingeführte CKM-Matrix.
Das Matrixelement $|V_{t,s}|$ ist hierbei weniger bekannt als andere Parameter der Matrix so dass deren Messung insbesondere zur Überprüfung des Standardmodells interessant ist.
Bisherige Messungen dieses Parameters konnten nur indirekt über Oszillationen im B-System durchgeführt werden.
Zukünftig interessant könnte eine direkte Messung über den Zerfall eines Topquarks in ein Strangequark sein.
Hierbei muss insbesondere der aus dem Strangquark entstehende Jet eindeutig identifiziert werden.
