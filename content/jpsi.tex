
\section[Die J/Psi-Entdeckung]{Die $J/\Psi$-Entdeckung}

\chapterauthor{Jan Langer, 16.11.2018}

\subsection{Historische Einordnung und Quarkmodelle}

Die Entdeckung des $J/\Psi$ im Jahre 1974 gilt als Nachweis der Existenz eines 4. Quarks, dem charm-Quark ($c$), sowie als allgemeine Bestätigung der Quark-Theorie.
Es handelt sich dabei um ein flavour-neutrales Meson, welches aus einem $c$ sowie einem $\overline{c}$ besteht.
Bereits im Jahr 1961 wurde der sogenannte Eightfold Way (Achtfache Weg) von Gell-Mann und Zweig vorgeschlagen, um die Teilchen mit der 1953 eingeführten Quantenzahl der Strangeness systematisch anzuordnen.
Hierbei werden die Mesonen, welche aus den zu diesem Zeitpunkt bekannten Quarks $u$, $d$ und $s$ bestehen, in einem Oktett sowie die Baryonen in einem Dekuplett angeordnet.
Dies entspricht einer Flavour $SU(3)$-Symmetrie, die annähernd zwischen den Quarks gültig ist.
Die Entdeckung des anhand diesem Modell postulierten $\Omega^-$-Quarks im Jahre 1964 festigte diese Theorie.


Im Jahr 1970 trat ein theoretisch unerwartetes Messergebnis im Zerfall von $K^0_\text{L}$-Mesonen auf, welches den sogenannten GIM-Mechanismus motivierte:
So konnte es bisher nicht erklärt werden, wieso der experimentell bestimmte Wert
\begin{align*}
	\frac{\Gamma\left( K^0_\text{L} \to \mu^+ \mu^- \right)}{\Gamma\left( K^0_\text{L}  \to \text{all}\right)} = \left(\num{9.1+-1.9}\right) \cdot \num{e-9}
\end{align*}
so gering ist.
Der von Glashow, Iliopoulus und Mainani beschriebe GIM-Mechanimus erklärt die Unterdrückung dieses Zerfalles durch eine destruktive Interferenz in Loop-Diagrammen erster Ordnung. Diese tritt durch die Einführung eines vierten Quarks, dem $c$-Quarks, auf.

Der Nachweis des $c$-Quarks über das $J/\Psi$ im Jahre 1974 bestätigt diese theoretische Erklärung, die bereits 1973 zur Erklärung der CP-Verletzung auf 3 Quarksfamilien erweitert wurde.

\subsection{Die Experimente am SLAC und BNL}
Das $J/\Psi$ wurde durch zwei unabhängige Experimente, dem Stanford Linear Accelerator Center (SLAC) und dem Brookhaven National Laboratory (BNL), entdeckt.
Die Entdeckung wurde auf einer gemeinsamen Pressekonferenz am 11. November 1974 offiziell bekannt gegeben.

Das Experiment am BNL, welches von Samuel Chao Chung Ting geleitet wurde, war ein fixed-target Experiment zur Suche nach Vektor-Mesonen und zur Untersuchung von deren Eigenschaften.
Der Beschleuniger konnte Protonen auf eine Energie von \SI{33}{\giga\electronvolt} beschleunigen, der Detektor war ein Zwei-Arm-Spektrometer.
Das Experiment am SLAC hingegen, unter Leitung von Burton Richter, war ein $e^+ e^-$-Speicherring mit dem Ziel, Hadronproduktion bei möglichst genauer Energie zu untersuchen.
Hierbei sollte insbesondere das Verhältnis $R$, d.h. der Anteil der Hadronproduktion bei Elektron-Positron Kollision, gemessen werden.
Der genutzte Detektor Mark1 war ein $4\pi$-Detektor, die maximale Schwerpunktsenergie im Schwerpunktssystem betrug $\SI{8}{\giga\electronvolt}$. 

\subsection{Ergebnisse der Experimente}

Der Peak in den Daten bei ca. $\SI{3.1}{\giga\electronvolt}$, welches der $J/\Psi$-Resonanz entspricht, wurde zunächst im August 1974 am BNL beobachtet, wobei zunächst auf eine Publikation verzichtet wurde.
Da es sich um einen $pp$-Collider handelt, war mussten bei den Analysen zur Trennung von Signal und Untergrund Spurrekonstruktionen durchgeführt werden.
Zudem wurden bei der Analyse der Daten  viele Gegenproben durchgeführt, beispielsweise durch die Änderung der Magnetströme, der Strahlintensität oder der Targetdicke.

Anfang November entdeckte auch das SLAC eine Erhöhung des Wirkungsquerschnittes $\sigma\left(e^+ e^- \to \text{hadrons}\right)$ im passenden Energiebereich, nachdem die Energie des Colliders auf die Resonanzenergie zurückgefahren wurde.
Feinere Messschritte ergaben eine deutliche Erhöhung des Wirkungsquerschnittes für hadronische Endzustände und somit den Nachweis der $J/\Psi$-Resonanz.

Während das SLAC-Experiment eine obere Grenze der Zerfallsbreite von $\Gamma \leq \SI{1.3}{\mega\electronvolt}$ angeben konnte, waren die Ergebnisse von BNL mit $\Gamma = 0$ kompatibel.
Die mit der schmalen Zerfallsbreits verbundene hohe Lebensdauer konnte mit den bisherigen Quarks nicht erklärt werden.
Die Deutung Tings, dass das $J/\Psi$ dementsprechend aus der Kombination $c\overline{c}$ besteht, bestätigte sich bereits nach einem Jahr.

Im Jahre 1976 erhielten beide Experimentatoren den Nobelpreis für die Entdeckung des $c$-Quarks.
Bereits wenige Tage nach der $J/\Psi$-Entdeckung folgte die Entdeckung vieler weiterer Resonanzen - diese Zeit wird historisch auch als Novemberrevolution bezeichnet.

\subsection{Diskussion}
In der Diskussion wurde gefragt, wie genau es zu der Novemberrevolution nach der $J/\Psi$-Entdeckung gekommen ist.
Nach der Publikation wurde genauer in den entsprechenden Energiebereichen gesucht, zusätzlich sind die Experimente im Laufe der Zeit besser geworden.

Zudem wurde der qualitative Unterschied zwischen einer $e^+ e^-$-Maschine und einem $pp$-Experiment besprochen.
Bei einem $e^+ e^-$ Collider handelt es sich um ein Präzisions-Experiment: Die Reaktionen sind sehr sauber und können durch einfaches Zählen der Endprodukte gut verstanden werden.
Dafür ist bei diesen Experimenten nur eine feste Schwerpunktsenerige vorhanden, die für neue Entdeckungen fortlaufend verändert werden muss.
Bei $pp$-Experimenten müssen die Endzustände aufgrund der hadronischen Reaktionen aufwendig rekonstruiert werden.
Dafür kann, aufgrund der an den Reaktion beteiligten Partonen, gleichzeitig ein großes Energiespektrum untersucht werden.