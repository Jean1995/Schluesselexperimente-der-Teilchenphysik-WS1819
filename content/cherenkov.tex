\section{Cherenkov-Detektoren}

\chapterauthor{Rune Dominik, 26.10.2018}

\subsection{Historische Entwicklung}
Die erste Beobachtung von Cherenkov-Strahlung, welche die Grundlage für Cherenkov-Detektoren bildet, lässt sich auf Marie Curie im Jahre 1910 zurückverfolgen.
Sie beobachtete erstmals das charakteristische blaue Licht, welches im Jahre 1934 durch den späteren Namensgeber Pawel Alexejewitsch Tscherenkow entdeckt wurde.
Er erhielt, zusammen mit Ilja Michailowitsch Frank und Igor Jewgenjewitsch Tamm, welche 1937 den Cherenkov-Effekt theoretisch erklärten, im Jahr 1958 den Nobelpreis.
Seitdem wird die Technik der Cherenkov-Detektoren, welche für den Teilchennachweis und die Teilchenidentifikation in der (Astro-)Teilchenphysik verwendet werden, stetig weiterentwickelt.

\subsection{Theoretische Grundlagen}
Der Cherenkov-Effekt beschreibt das Auftreten von elektromagnetischer Strahlung, wenn überlichtschnelle, geladene Teilchen ein Medium durchqueren.
Besitzt das Teilchen die Geschwindigkeit $\beta c$ und beträgt die Lichtgeschwindigkeit im Medium $\sfrac{c}{n}$, wobei $n$ den Brechungsindex im entsprechenden Medium beschreibt, so gilt für den Winkel zwischen Teilchenbahn und Emission des Cherenkov-Lichts die Formel
\begin{equation}
	\cos{\theta} = \frac{1}{n \beta}.
\end{equation}
Das Spektrum des emittierten Lichts, welches durch die Frank-Tamm-Gleichung beschrieben wird, besitzt hierbei eine $\lambda^{-2}$-Abhängigkeit.

Um Detektoren in ihrer Effizienz vergleichen zu können, kann die Detektoreffizienz $\epsilon$ verwendet werden, welche den Anteil der tatsächlich gemessenen an den theoretisch messbaren Cherenkov-Photonen beschreibt.

\subsection{Experimenteller Aufbau von Cherenkov-Detektoren}
Ein grundlegender Cherenkov-Detektor besteht aus einem Radiator, in welchem durchgehende Teilchen Cherenkov-Licht emittieren sowie einer Detektoreinheit, welche das Licht detektiert. Zudem wird eine Anordnung, beispielsweise aus Spiegeln, benötigt, welche das Licht vom Radiator auf den Detektor lenkt.
Die exakte Ausführung dieses Konzeptes ist detektorabhängig, wobei im Folgenden das Konzept der Schwellwertdetektoren, der differentiellen Detektoren sowie der RICH-Detektoren erläutert.

\subsubsection{Schwellwertdetektoren}