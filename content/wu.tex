
\section{Wu-Experiment}

\chapterauthor{Inga Höfmann, 02.11.2018}

\subsection{Historische Entwicklung}
Das Wu-Experiment beschäftigt sich mit der Parität, welche als Operator eine Inversion in allen Raumkoordinaten beschreibt. Es gilt
\begin{align*}
	\hat{P} \vec{r} &= -\vec{r}, &
	\hat{P} \vec{L} &= \vec{L}.
\end{align*}
Die Parität wurde erstmals im Jahre 1927 als Quantenzahl und somit intrinsische Eigenschaft eines Teilchen eingeführt und galt zunächst für alle Wechselwirkungen als Erhalten.
Zweifel an der Parität als Erhaltungsgröße kamen im Jahr 1954 mit dem sogenanten $\tau-\theta$-Puzzle auf:
Die damals als $\tau-$ und $\theta$-Meson bekannten Teilchen besaßen identische Massen, Ladungen und Lebensdauern.
Unteschiede ergaben sich lediglich in den Zerfallsprodukten:
Während das $\tau^+$-Meson in den Endzustand $\pi^+ \pi^+ \pi^-$ (Parität $-1$) zerfiel, war der Endzustand des $\theta^+$-Mesons $\pi^+ \pi^0$ (Parität $+1$).
Als Erklärung dieses Puzzles existierte die Möglichkeit, dass es sich bei beiden Mesonen um das gleiche Teilchen handelt und die Parität in diesem Fall verletzt ist.
Diese Vermutung wurde, zusammen mit Vorschlägen für einen experimentellen Aufbau, im Jahre 1957 von Tsung-Dao Lee und Chen Ning Yang in einem Paper veröffentlich, wofür sie 1957 den Nobelpreis erhielten.
Das dazugehörige Experiment wurde von der Physikerin Chien-Shiung Wu im Jahre 1956 innerhalb von neun Monaten realisiert.

\subsection{Idee des Wu-Experimentes}
Die Idee des Wu-Experimentes ist es, eine Paritätsverletzung in der schwachen Wechselwirkungs nachzuweisen.
Um einen Interferenzterm mit paritätserhaltenden und paritätsverletzenden Anteilen untersuchen zu können muss ein Pseudoskalar als Messgröße genutzt werden.
Hierzu bietet es sich an, die Winkelverteilung $I\left(\theta \right)$ von $e^-$ in einem $\beta$-Zerfall zu betrachten. 
Es ergibt sich ein Parameter $\alpha$, welcher die Asymmetrie in der Winkelverteilung beschreibt, wodurch $\alpha = 0$ für Paritätserhaltung bzw. $\alpha \neq 0$ für Paritätsverletzung gilt.

Das Wu-Experiment ist so konzipiert, dass es rein qualitative Aussagen über den Parameter $\alpha$ treffen kann.
Als Probe wird $\ce{^{60}_{27}Co}$ verwendet, welches dominant unter Aussendung eines $e^-$ und eines $\bar{\nu_e}$ (d.h. unter der schwachen Wechselwirkung) in $\ce{^{60}_{28}Ni^{*}}$ zerfällt. Dieses sendet widerum zwei energetisch charakteristische Photonen aus um in einen nicht angeregten Zustand zu gelangen.
Zu beachten ist, dass $\ce{^{60}_{27}Co}$ einen Spin von $5$ und $\ce{^{60}_{28}Ni^{*}}$ einen Spin von $4$ besitzt